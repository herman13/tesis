Recolect and distribute information after a disaster can be complex because of
the damages caused to the comunication infrastructure and the different
behaviours of the users of the network. Current solutions exploit mobile devices
using protocols for Delay Tolerant Networks (DTN) that allow devices to send and
receive information with intermitent conectivity. However, most DTN protocols
have high energy consumption, which is a scarce resource in mobile devices in
desaster scenarios. One of the factors that increases energy consumption in
these protocols is the relay of messages, that can be reduced using more than
one protocol in the same network.



In this thesis, two new DTN protocols are proposed for disasters scenarios with
the goal of reducing energy consumption of devices by using more than one
protocol in the network, i.e., each node exchanges messages following differents
routing rules depending in their mobility and performance of the network
optimizing the avalaible resources. Moreover, a secure DTN protocol is presented
that adds a security layer to mitigate denial of service attacks. 


Using simulations to compare the proposed protocols with the state of the art,
it is found that the proposed protocols are capable to reduce the energy
consumption in the network. In the case of the security protocol, it is capable
of mitigate a denial of service attack.  Finally, in order to analyse the
protocols in a city of Chile, a new mobility model based in the city of
Valparaíso is created using information of the National Emergency Office
(ONEMI).


This work is part of the project DICYT-USACH 061419RO \textit{Efficient data
recollection for mobile devices.} 
