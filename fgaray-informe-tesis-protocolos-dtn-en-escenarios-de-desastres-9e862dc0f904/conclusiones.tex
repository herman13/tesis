% Hablar que los protocolos seleccionados se pueden usar en otros contextos


En este trabajo se propusieron dos protocolos híbridos para DTN en situaciones
de desastres orientados a la reducción del consumo de energía, uno asignando de
forma estática los protocolos a los nodos y otro que es capaz de seleccionar
automáticamente el protocolo a ejecutar dependiendo del desempeño de la red.
Además, se propuso un nuevo protocolo orientado a la seguridad en escenarios de
desastres que permite el intercambio de métricas de la red con no repudio en
escenarios de desastres.

Para permitir la correcta evaluación del protocolo híbrido dinámico se propuso
además una adaptación del modelo \textit{Post-Disaster Mobility Model} a la
ciudad de Valparaíso utilizando información proporcionada por la Oficina
Nacional de Emergencias sobre los movimientos que ocurren en un desastre natura
en Chile.


\seccion{Objetivos del trabajo}

En cuanto a los objetivos específicos se puede decir que:

\begin{enumerate}
    \item \textbf{Elaborar un estado del arte sobre redes tolerantes a disrupciones,
      modelos de movilidad y modelos de energía para dispositivos móviles}: Fue
      cumplido en el \ref{chp:arte} donde se presentaron los últimos trabajos
      relacionados con redes DTN.
    \item \textbf{Adaptar el modelo de movilidad \textit{Post-Disaster Mobility
    Model} (PDM) \cite{uddin_post-disaster_2009}  a Chile}: Fue cumplido en el
    \ref{chp:movilidad} donde se presentaron los parámetros y movimientos que se
    deberían realizar en PDM para asemejarse más a un escenario chileno.
    \item \textbf{Diseñar un protocolo para redes tolerantes a
      disrupciones que se adapte las decisiones de enrutamiento en base a la
      movilidad de los dispositivos móviles participantes}: Fue cumplido en el
      \ref{chp:pasado} donde se hizo una exploración de como un protocolo
      híbrido podría mejorar el consumo de energía en un dispositivo móvil,
      luego en el \ref{chp:dinamico} se presenta la forma en la que un protocolo
      puede adaptarse y tomar la decisión de que protocolo utilizar en una red
      y escenario. Finalmente, como el diseño de un protocolo de comunicación debe
      abordar el problema de la seguridad, en el \ref{chp:seguridad} se presenta
      un protocolo que permite el no repudio de las métricas de los protocolos y
      mitigar ataques de denegación de servicio.
    \item \textbf{Implementar el protocolo en el simulador de redes tolerantes a
      disrupciones \textit{The One}}: Se cumplió en el \ref{chp:pasado},
      \ref{chp:dinamico} y \ref{chp:seguridad}, donde cada uno de los protocolos
      fue implementado en el simulador.
    \item \textbf{Evaluar el rendimiento del protocolo híbrido respecto a los protocolos
      del estado del arte en diferentes modelos de movilidad}: Fue cumplido en el
      \ref{chp:pasado}, \ref{chp:dinamico} y \ref{chp:seguridad} en sus
      respectivas secciones acompañando los resultados de análisis. Además, el
      \ref{chp:movilidad} permite evaluar el desempeño del protocolo híbrido
      dinámico en un escenario más parecido al chileno.
\end{enumerate}


El objetivo general "Crear un protocolo de redes tolerante a disrupciones  para
dispositivos móviles en escenarios de desastre" se cumplió completamente con la
creación del protocolo híbrido estático, el protocolo híbrido dinámico y el
protocolo de seguridad, los tres para escenarios de desastres y diseñados para
reducir el consumo de energía.


\seccion{Problemas encontrados}

El tamaño de las simulaciones que se querían realizar funcionan bien para
ciertos protocolos DTN y modelos de movilidad menos exigentes los tiempos de
simulación son manejables, de entre $1$ a $3$ horas para \textit{Working Day
Mobility Model} \cite{ekman_working_2008}, pero cuando se quieren hacer
simulaciones más complejas como con \textit{Post-Disaster Mobility Model}
\cite{uddin_post-disaster_2009}, algunas simulaciones pueden llegar a tardar
$24$ horas. Este excesivo tiempo de simulación hace difícil depurar los
protocolos implementados y utilizar una metodología de desarrollo iterativa. Una
solución a este problema habría sido el uso de \textit{Mocking} que permite
imitar el comportamiento del simulador en una escala más pequeña para realizar
mejores pruebas unitarias en un tiempo menor. A pesar de la posibilidad de crear
clases \textit{Mock} para probar los protocolos, aún persiste el problema del
tiempo total de las simulaciones que limitan la cantidad de experimentos que se
puedan realizar, especialmente en cuanto al tamaño de la simulación y cantidad
de participantes.



Así mismo, se trató de utilizar simulaciones paralelas de granularidad fina para
reducir los tiempos de simulación al paralelizar secciones del código que
resultaban costosas de ejecutar. Luego de invertir un tiempo significativo en
ese proyecto se llegó a la conclusión que la arquitectura del simulador requería
de mayores modificaciones para permitir la paralelización y seguir manteniendo
el determinismo de los resultados.



\seccion{Trabajo futuro}


Una mejora que podría ser aplicada al protocolo híbrido dinámico es agregar
métricas y contextos adicionales dependiendo de la función objetivo que se
quiera optimizar. Explorar métricas como la latencia para el caso de que los
mensajes tengan prioridades o propósitos específicos como distribuir rápidamente
los mensajes por la red desde autoridades hacia personas también es algo que
debería probarse.


El modelo de movilidad de Valparaíso (VMM) basado en \textit{Post-Disaster
Mobility Model} no ha sido comparado con el movimiento real, solamente se ha
utilizado información de documentos y fuentes oficiales de la Oficina Nacional
de Emergencias, por lo que en el futuro se podrían recolectar trazas mediante el
uso de dispositivos móviles cuando se hacen los simulacros de evacuación y poder
mejorar el modelo de movilidad.

El protocolo de seguridad, como fue presentado, es independiente de los
protocolos híbridos por lo que sería interesante mezclar ambas soluciones en una
sola implementación y realizar simulaciones para ver su comportamiento como
conjunto.
