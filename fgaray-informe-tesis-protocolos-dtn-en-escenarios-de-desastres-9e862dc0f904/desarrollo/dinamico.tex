En trabajos previos, como el de \textit{MetaRouter}
\cite{paper_evaluacion_nosotros}, se explora la viabilidad de ejecutar más de un
protocolo en una red DTN mostrando que es posible reducir el consumo de energía
dentro de la red. Utilizando este concepto se propone una versión dinámica que
permita seleccionar automáticamente el protocolo más adecuado en base a un
criterio establecido sin tener que asignar a priori un protocolo.

El protocolo está construido utilizando el lenguaje de programación Java y
Scala, este último un lenguaje de programación funcional y orientado a objetos.
La mezcla entre ambos paradigmas permite una mayor expresividad y abstracción de
tareas comunes de programación y la reutilización de código mediante la herencia
de clases del simulador.

Para evitar confusiones de nombres, el \textbf{protocolo híbrido dinámico} es
aquel meta protocolo que está siendo ejecutado por el dispositivo y que es el
encargado de tomar la decisión de que protocolo del estado del arte ejecutar,
mientras que se va a utilizar \textbf{protocolo del estado del arte} o
\textbf{protocolo actual} para referirse a aquellos protocolos que pueden ser
seleccionados para la transmisión de mensajes. Para realizar las simulaciones de
este protocolo se van a utilizar los siguientes protocolos del estado del arte:
\syw{} \cite{spyropoulos_spray_2005}, \syf{} \cite{spyropoulos_spray_2007},
\epidemic{} \cite{amin_vahdat_epidemic_2000}, \maxprop{}
\cite{burgess_maxprop:_2006} o \prophet{} \cite{lindgren_probabilistic_2003}.


A continuación se describe un protocolo híbrido dinámico general que puede
utilizado en distintas redes sin importar cual es el propósito del protocolo
(maximizar tasa de entrega, minimizar \overhead, minimizar latencia, etc), luego
se presenta un protocolo híbrido dinámico para escenarios de desastres basado en
los conceptos de \textit{MetaRouter}.  Este capítulo se divide en una sección de
diseño donde se explica el funcionamiento interno del protocolo, luego una
sección de evaluación donde se compara el protocolo con otros del estado del
arte en distintos escenarios de movilidad y finalmente se presentan las
conclusiones, problemas y posibles mejoras que se puedan realizar al protocolo.


\seccion{Diseño del protocolo híbrido dinámico general}

El protocolo presentado en esta sección permite la creación de una familia de
protocolos híbridos dinámicos con distintas funciones objetivos a minimizar o
maximizar para DTN.

Un nodo es un dispositivo móvil o estático que contiene almacenamiento,
capacidad de procesamiento y conectividad inalámbrica compatible con los nodos
del resto del escenario como pueden ser \textit{WiFi} o \textit{Bluetooth}.  Se
entiende por contexto a una n-tupla $(c_1, c_2, ..., c_n)$ de valores numéricos
que caracterizan el ambiente en el cual se encuentra un nodo. Las métricas de
evaluación son aquellos valores que ayudan a identificar como un nodo se
comporta, además, son locales al nodo y no globales como las métricas entregadas
por \theone{} al finalizar la simulación.

El protocolo híbrido dinámico general es capaz de utilizar el contexto actual en
el cual se está ejecutando y la retroalimentación de la red en forma de métricas
para seleccionar aquel protocolo que maximiza o minimiza una función objetivo.
De esta forma, los nodos son capaces de automáticamente seleccionar aquel
protocolo más adecuado, de acuerdo a su función, sin requerir de conocimiento
previo debido a que este es recolectado de las interacciones de los nodos.




\figura{Algoritmo del protocolo híbrido dinámico general.}{
\begin{algorithm}[H]
     \SetAlgoLined
     \While{se ejecuta el protocolo}{
       \If{se encuentra con un nodo vecino}{
         agregar métricas y contextos locales a los mensajes locales en el \textit{buffer}\;
         intercambiar mensajes de acuerdo al protocolo del estado del arte seleccionado\;
         extraer métricas y contextos de los mensajes recibidos\;
         actualizar métricas locales\;
         actualizar contextos locales\;
         \If{ha vencido el tiempo de cambio de protocolo}
         {
            seleccionar el protocolo del estado del arte adecuado para el contexto actual;
         }
       }
     }
\end{algorithm}
}{fig:algoritmo}



En la \ref{fig:algoritmo} se muestran los pasos que sigue el protocolo dinámico
general cuando se está ejecutando dentro de un dispositivo móvil. Al encontrarse
el nodo con un vecino, se utiliza la interfaz de red para intercambiar
los mensajes de acuerdo a las reglas del protocolo seleccionado. Cada uno de los
mensajes contiene información de cada uno de los nodos por los que ha sido
transmitido: protocolo ejecutado, métricas del protocolo y contexto en el cual se ha
ejecutando el protocolo. Cuando se recibe un mensaje desde la red, se le extrae
esta información para ser almacenada hasta el momento en el cual el nodo debe
seleccionar un nuevo protocolo. El protocolo híbrido dinámico va guardando pares
de información de la forma $(c, t)$ que pueden ser representadas en un gráfico
con el eje $x$ como el eje de los contextos $c$ y el eje $y$ como el eje de las
métricas $t$, debido a que el desempeño de un protocolo (métrica) es dependiente
del contexto del nodo en el cual se está ejecutando. En otras palabras, el
contexto corresponde a la variable independiente y la métrica a la variable
independiente de un gráfico.


El gráfico creado a partir de contextos y métricas permite estimar cual va a
ser el desempeño de un protocolo en un contexto en particular debido a que es
posible interpolar valores intermedios y obtener la métrica en ese punto.  La
técnica a utilizar depende de la calidad de los contextos y métricas que se
encuentren disponibles: si se tienen contextos y métricas que se puedan calcular
con precisión entonces una buena interpolación puede entregar un buen estimado,
en caso contrario, si son poco exactos los contextos y métricas, una
interpolación más complicada no va a entregar un buen valor estimado debido a
que está trabajando sobre valores imprecisos. Utilizar una interpolación más
simple en caso de tener métricas y contextos estimados puede permitir ahorrar
tiempo de computación y consumo de energía en el dispositivo.

\figura{Gráfico de muestra de la información recolectada dentro de un protocolo
híbrido dinámico general.}
{\pgfplotsset{cycle list/Dark2}
\renewcommand{\colorscheme}{Dark2}

\begin{tikzpicture}
\begin{axis}[
    xlabel={Contexto},
    ylabel={Métrica},
    xmin=0, xmax=10,
    ymin=0, ymax=1,
    legend style = {
      legend pos = outer north east,
    },
    width=0.7\textwidth,
	  scatter/classes={
      a={mark=square*,blue},
      b={mark=triangle*,red},
      c={mark=o,draw=black}
    }
]

\legend{Protocolo 1,Protocolo 2,Protocolo 3}

\addplot+[
  scatter,
  very thick,
  color=blue,
  mark=square*
  ]
    table[meta=label] {
      x y label
      1 0.2 a
      5 0.3 a
      6 0.5 a
    };


\addplot+[
  scatter,
  only marks,
  very thick,
  color=red,
  mark=triangle*
  ]
    table[meta=label] {
      x y label
      3 0.4 b
      4 0.5 b
      7 0.1 b
    };


\addplot+[
  scatter,
  only marks,
  very thick,
  color=black,
  mark=o
  ]
    table[meta=label] {
      x y label
      3 1 c
      10 0.1 c
    };

 
\end{axis}
\end{tikzpicture}

}{fig:contextos-grafico}

La \ref{fig:contextos-grafico} muestra la posible información recolectada por un
protocolo híbrido dinámico, donde en el eje $x$ se representa un contexto y el
eje $y$ una métrica, para tres protocolos. Además, se muestra como un protocolo
híbrido dinámico utiliza la interpolación de rectas para estimar valores
intermedios de las métricas. Este tipo de estructuras se debe construir para
cada uno de los tipos de pares $(c, t)$ que se implementen en el protocolo
híbrido dinámico general.

Permitir que en cada instante de la ejecución del protocolo híbrido dinámico 
seleccione un nuevo protocolo sería ineficiente dado que la cantidad de
información nueva que recibe el protocolo con solo una interacción no es
suficiente para llegar a una conclusión sobre que protocolo es el más adecuado,
es decir, no se debería tomar en cuenta las métricas y contextos de un solo nodo
en un solo instante de tiempo, se necesita recolectar más información. Es por
esta razón que se utiliza el tiempo de cambio de protocolo, para limitar la
cantidad de selecciones de protocolos que se pueden hacer en un instante de
tiempo definido. Un valor alto del tiempo de cambio de protocolo implica menos
selecciones de protocolos del estado del arte y un valor pequeño implica más
selecciones de protocolos lo que hace que el protocolo dinámico sea más agresivo
en cuanto a su optimización.
  

El cálculo de contextos y métricas debe ser realizado dentro de una ventana
deslizante de tiempo en la cual la historia pasada tiene un nulo o bajo peso
respecto a la historia más reciente. Esto permite que los contextos y métrica
representen el estado actual del protocolo y del nodo, y no un pasado que puede
no tener relación con el momento presente de la red. 


La selección del protocolo utiliza una función objetivo $f(t_1..t_m)$ donde
$(t_1..t_m)$ son las métricas utilizadas en el protocolo híbrido dinámico.
Utilizando el contexto local del dispositivo móvil que quiere seleccionar un
nuevo protocolo y la interpolación de los puntos del gráfico de contexto versus
métricas, es posible estimar la métrica que va a obtener un protocolo en el nodo
actual. Para cada uno de los protocolos que estén disponibles para ser
seleccionados, se estiman las métricas mediante interpolación para el contexto
del nodo y se evalúan en la función $f$. Cuando cada protocolo tiene su
evaluación, se selecciona aquel protocolo que maximice o minimice $f$, dependiendo
del objetivo, como el protocolo que debería ser ejecutado por el nodo.

Algunas métricas que se pueden utilizar dentro del protocolo son: tasa de
entrega, copias de los mensajes, latencia de la red, retransmisiones de mensajes
y tiempo promedio dentro de los \textit{buffers}. En cuando a los contextos,
algunos de estos pueden ser: densidad de nodos vecinos, tiempo promedio de
contacto, prioridad de los mensajes y diferencias de recursos de los
dispositivos (rango y velocidad de la conexión inalámbrica).


La distribución de las métricas y contextos se realiza sobre los mismos mensajes
de usuario que la red está tratando de hacer llegar a los destinos,  mientras
más mensajes sean transmitidos en la red mejor va a ser la distribución de
métricas y contextos entre los nodos. En cada uno de los saltos que realiza un
mensaje de usuario, el nodo que está enviando el mensaje a su vecino agrega sus
contextos y métricas actuales junto con el protocolo del estado del arte que fue
utilizado para realizar la transferencia.

El espacio adicional necesario para almacenar la información de contextos y
métricas en un mensaje se define en función de la cantidad de contextos y
métricas que utiliza el protocolo dinámico. Asumiendo un identificador entero
para los protocolos del estado del arte, se necesitarían $\lceil \log_2(p)
\rceil$ bits para identificarlos en un mensaje, donde $p$ es la cantidad de
protocolos del estado del arte utilizados. Además, para los contextos y métricas
se necesitarían un total de $4*c*t$ bytes, asumiendo un número de coma flotante
de 4 bytes, donde $c$ es la cantidad de contextos que implementa el protocolo
híbrido dinámico y $t$ es la cantidad de métricas implementadas. Debido a que en
cada salto se agrega información de los contextos y métricas, el total de
espacio requerido depende de la cantidad de saltos $h$ que se realicen, quedando
finalmente el espacio adicional requerido en $(\lceil \frac{\log_2(p)}{8} \rceil
+ 4*c*t)*h$ bytes. En el peor caso que un mensaje sea transmitido por todos los
nodos de la red se puede considerar $h = n$ donde $n$ es el total de nodos de la
red.


Utilizando los pasos presentados del protocolo híbrido dinámico es posible crear
una familia de protocolos, cada uno con una serie de contextos, métricas y
funciones objetivos con el propósito de optimizar algún aspecto de la red. Por
ejemplo, un protocolo que quiera centrarse en entregar mensajes rápidamente por
la red utilizaría como métrica la latencia de los mensajes entregados y los
tiempos de espera en los \textit{buffers} de los nodos. 









%%%%%%%%%%%%%%%%%%%%%%%%%%%%%%%%%%%%%%%%%%%%%%%%%%%%%%%%%%%%%%%%%%%%%%%%%%%%%%%%

\seccion{Diseño de un protocolo híbrido dinámico para escenarios de desastres}

Debido a que la vida de una DTN depende de la disponibilidad energética que
tengan los nodos de la red, un protocolo que se quiera ejecutar en un escenario
de desastre debe tratar de minimizar el consumo de energía proveniente de la
comunicación de mensajes entre dispositivos dado que la energía eléctrica
utilizada para recargar los dispositivos puede no estar disponible si la
infraestructura de la ciudad ha sido dañada como en un terremoto.

A continuación se presenta un protocolo híbrido dinámico para escenarios de
desastres que minimiza el consumo de energía relacionado con el \overhead{} de
mensajes de acuerdo a la ecuación (\ref{eq:energia}). Este protocolo híbrido
dinámico utiliza como contexto la densidad de nodos en el tiempo y los tiempos
promedio de contacto, los cuales permiten caracterizar la movilidad que tienen
los nodos dentro de la red: un nodo con una alta movilidad va a tener un tiempo
promedio de contacto diferente que uno con una baja movilidad, de igual forma
con la densidad de nodos. Se decide por utilizar el movimiento de los nodos como
contextos debido a que en el \ref{chp:pasado} se demostró que era importante
asignar un protocolo distinto a los nodos de acuerdo a como se movían en el
escenario.

Dado que el objetivo de este protocolo es la reducción del consumo de energía,
se van a utilizar dos métricas para evaluar el desempeño de la red: tasa de
entrega y copias de mensajes en el tiempo. La razón de utilizar las copias de
mensajes es debido a que estas están relacionadas con el consumo de energía como
se puede ver en la ecuación (\ref{eq:energia}). En cuanto a la tasa de entrega,
esta también se encuentra relacionada con el consumo de energía debido a su
relación con el \overhead{} de mensajes, a mayor copia de mensajes en la red,
entonces mejor tasa de entrega, asumiendo que los \textit{buffers} de los
dispositivos miembros de la red no se saturan. La mezcla de las dos métricas
evita situaciones donde un nodo podría seleccionar un protocolo del estado del
arte que tenga cero copias de mensajes dado que en ese caso también tendría una
tasa de entrega de cero.

En la \ref{fig:algoritmo-desastre} del \ref{apendice:dinamico-alg} se pueden ver
los pasos que sigue el protocolo híbrido dinámico en cuanto al cálculo de
métricas y contextos e intercambio de mensajes. Esta es una versión menos
genérica que la de la \ref{fig:algoritmo}.

Finalmente, el protocolo híbrido dinámico para escenarios de desastres debe
tomar estas dos métricas en un contexto particular y equilibrarlas de manera tal
de reducir el consumo de energía pero sin perjudicar la tasa de entrega en gran
medida.






\subseccion{Métricas del protocolo híbrido dinámico}
\customlabel{sub:metricas}{Sección \arabic{chapter}.\arabic{section}.\arabic{subsection}}

A continuación se explica la forma en que las métricas son calculadas dentro del
protocolo híbrido dinámico para escenarios de desastres.

La tasa de entrega puede ser calculada de acuerdo a la ecuación
(\ref{eq:delivery-dinamico}), donde $d$ es la cantidad de mensajes entregados y
$c$ es la cantidad de mensajes creados en la red. Esta es una métrica que se
calcula de manera global, pero es posible estimarla localmente en los nodos. La
estimación no va a ser exacta al valor global dado que la comparación se va a
hacer con otros valores locales calculados en otros nodos y no con el valor
global. El valor de $c$ es sencillo de obtener debido a que es la cantidad de
mensajes que ha recibido el nodo de otros más la cantidad de mensajes que se han
creado en el mismo nodo. Si este nodo es capaz de enviar todos estos mensajes a
sus destinos, entonces el nodo tendría una tasa de entrega de $1$, pero en el
caso que no logre entregar ninguno, entonces tendría un valor de tasa de entrega
de $0$. Luego, se obtiene el valor de $d$, el cual es la cantidad de mensajes
que han sido entregados con éxito en sus destinos. Es necesario entonces que
cuando un mensaje llegue al destino, se notifique al resto de los nodos de la
red de su llegada, para que puedan actualizar sus valores de $d$ si es que han
participado de la transmisión de ese mensaje. La solución más directa es la de
crear un nuevo mensaje y enviarlo de forma epidémica a la red. El problema de
esta primera aproximación es que se generaría un mayor número de copias de
mensajes lo que implicaría un mayor consumo de energía relacionado con la
comunicación. Una alternativa es utilizar los mismos mensajes de la red como
portadores de meta información para distribuir esta información lo que reduce el
\overhead{} al costo de que existan partes de la red que no se actualicen,
perdiendo precisión. Además, existe una paradoja al utilizar
\textit{piggybacking} para notificar de los mensajes entregados en la red debido
a que esta técnica funciona mejor mientras más mensajes hay circulando por la
red dado que hay mayor probabilidad de distribuir la información, pero debido a
que el objetivo del protocolo híbrido dinámico es reducir el consumo de energía
de comunicación, va a tender a reducir el \overhead{} de mensajes limitando la
eficiencia con la cual se distribuyen los mensajes en la red.


\begin{equation}
    \label{eq:delivery-dinamico}
    t = \frac{d}{c}
\end{equation}

Cuando un mensaje llega a un nodo que es el destino, el identificador del
mensaje y los identificadores de los nodos y protocolos utilizados en cada salto
se guardan para ser distribuidos por la red utilizando
\textit{piggybacking}. Cada vez que otros mensajes son retransmitidos por este
nodo, se le agregan los identificadores de los mensajes que han llegado de
manera de que al ser distribuidos por la red puedan notificar a los nodos de los
mensajes que han logrado llegar al destino.

La información adicional que se incluye en los mensajes es en el peor caso $m*h$
entradas, donde $m$ es la cantidad de mensajes totales creados en la red y $h$
es la cantidad de saltos máxima que puede realizar un mensaje, que en el peor
caso sería $h = n$ donde $n$ es el total de nodos en la red. Se puede expresar
la información adicional en bytes como $(\lceil \frac{\log_2(m)}{8} \rceil +
\lceil \frac{\log_2(n)}{8} \rceil) * m * h$ bytes. Es posible reducir este
\overhead{} si se utilizan estructuras de datos alternativas como un
\textit{Bloom Filter}. Un \textit{Bloom Filter} es una estructura de datos
probabilística que tiene falsos positivos, lo que provocaría sobrestimar la tasa
de entrega. Es posible reducir la probabilidad de falsos positivos en hasta un
$1\%$ si es que se utilizan al menos $10$ bits por elemento, es decir, utilizando
esta estructura un mensaje con \textit{piggybacking} necesitaría $10*m*h$ bits o
$2*m*h$ bytes para almacenar los mensajes entregados
\cite{DBLP:conf/esa/BonomiMPSV06}.




El \overhead{} de copias de mensajes se calcula de manera indirecta sin tomar en
cuenta la cantidad de mensajes que han llegado al destino, esto debido a que se
quiere que esta métrica sea una forma de medir la cantidad de copias que aporta
el protocolo dentro de la red. Mientras se ejecuta un protocolo, cada vez que se
realiza una copia de un mensaje, esta es agregada a un contador donde se guardan
la cantidad de copias y cuando se cambia el protocolo este contador es guardado
y reemplazado por el que le corresponde al protocolo recientemente seleccionado
, así la historia pasada del protocolo continua siendo utilizada y no se borra.
El problema de utilizar un contador de esta forma es que puede perjudicar a
protocolos que se ejecuten más que otros: un protocolo que se ejecute más
tiempo va a generar una cantidad de copias mayor que otro que se ejecute menos,
a pesar que puede que el segundo protocolo genere más copias si se llega a
ejecutar el mismo tiempo que el primero. Para evitar esto es que se divide la
cantidad de copias realizadas por un protocolo por el tiempo que ese protocolo
ha sido utilizado, razón por la cual esta métrica representa la cantidad de
copias realizadas por el protocolo en el tiempo. La ecuación
(\ref{eq:overhead_tiempo}) muestra la forma de calcular esta métrica.

\begin{equation}
  Overhead = \frac{Copias}{Tiempo}
  \label{eq:overhead_tiempo}
\end{equation}


El máximo valor que puede tomar el \overhead{} depende del modelo de movilidad y
del protocolo, por lo tanto es necesario normalizar entre $0$ y $1$ para aplicar
de forma correcta la función objetivo. El valor máximo de la normalización se
encuentra mediante simulaciones.



\subseccion{Contextos}


Los contextos implementados en el protocolo son la densidad de nodos vecinos en
el tiempo y el tiempo promedio de contacto.

La densidad de nodos vecinos en el tiempo se calcula de acuerdo a la cantidad de
encuentros que tiene el nodo dentro de un intervalo o ventana de tiempo. Cada
encuentro es guardado como una entrada en una cola FIFO (\textit{first in, first
out}), la cual contiene información sobre el tiempo de simulación en el cual
cada encuentro fue efectuado. Para calcular la métrica, primero se eliminan las
entradas que hayan salido de la ventana de tiempo y por lo tanto son muy
antiguas como para cuantificar de manera adecuada el contexto actual. Luego se
cuenta la cantidad que elementos que aún se encuentran en la cola y este número
es dividido por la cantidad de segundos de la ventana, generando así una tasa de
nodos vecinos en el tiempo. Un valor alto de este contexto indica que existe una
gran densidad de nodos vecinos y por lo tanto el nodo tiene una alta
conectividad.  En la ecuación (\ref{eq:encuentros}) se puede ver la forma de
calcular este contexto.

\begin{equation}
  Densidad = \frac{Cantidad \; encuentros}{Tiempo \; ventana \; contextos}
  \label{eq:encuentros}
\end{equation}


El tiempo promedio de contacto mide de una forma indirecta la rapidez de un
nodo y por lo tanto es un indicativo de que nodos pueden acceder a distintas
partes de la red en un corto tiempo. Cada vez que hay un nuevo encuentro se
guarda el tiempo actual de la simulación y se genera una entrada en una tabla.
Luego cuando la conexión es cerrada con el nodo vecino, es decir, el encuentro
termina, se obtiene de la tabla el tiempo en el cual se encontró inicialmente al
nodo y se calcula la diferencia con el tiempo actual de la simulación. Esta
diferencia es el tiempo que duró ese contacto, el cual es guardado con el tiempo
actual de la simulación en una cola FIFO. Cuando se quiere calcular el promedio
de los tiempos de encuentros, se eliminan aquellos encuentros antiguos de la
misma forma que con la densidad de encuentros y se calcula el promedio entre
todos los tiempos. Un valor promedio alto indica que los encuentros tienden a
ser largos y por lo tanto el nodo se debe mover lento, mientras que un valor
bajo indica que el nodo se mueve rápidamente. A diferencia del contexto de
densidad de nodos, el promedio de los tiempos de contactos se puede estimar sin
que sea necesario terminar el encuentro y, por lo tanto, puede cuantificar
cuando los nodos se quedan quietos en un punto. En la ecuación (\ref{eq:tiempos})
se muestra la forma de calcular este contexto donde cada encuentro $i$ tiene un
tiempo de contacto que es dividido por el total de encuentros $n$.


\begin{equation}
  Tiempo = \frac{\sum^n_i Tiempo\;contacto_i}{n}
  \label{eq:tiempos}
\end{equation}


Ambos contextos dependen del rango de la tecnología de comunicación inalámbrica
que se esté utilizando, debido a que los encuentros son detectados con el
descubrimientos de nuevos pares en la interfaz de red.



\subseccion{Selección del protocolo a utilizar}

Los nodos participantes de la red generan métricas para el protocolo del estado
del arte que están ejecutando en el momento y contexto actual. Esta información
es agregada como datos extra a la información de usuario original del mensaje.
Debido a que los mensajes saltan de nodo en nodo a lo largo de la red mientras
buscan sus destinos es que los mensajes van a tener una lista de protocolos,
métricas y contextos por cada uno de los saltos que ha realizado.



Al llegar un mensaje con información de evaluaciones a un nodo, el primer paso
es extraer la información adicional para luego almacenar en el \textit{buffer}
local el mensaje a la espera de ser transmitido a los vecinos. Esta información
de evaluaciones es almacenada en una estructura que se asemeja a un gráfico de
evaluación de métricas, donde se tienen variables independientes o contextos que
van a alterar el comportamiento de variables dependientes o métricas. Esta
estructura va a permitir automatizar el proceso de la toma de decisiones que
utiliza cuando se evalúa un protocolo, es decir, seleccionar el protocolo que
presente el mejor comportamiento para una situación en particular.


Esta estructura puede ser vista a través de un ejemplo en la
\ref{fig:contextos-grafico}, donde hay tres protocolos: protocolo 1 y 2 tienen 3
evaluaciones  y el protocolo 3 tiene dos evaluaciones. Cada uno de los puntos
corresponde al desempeño de un protocolo en un contexto en particular. Por
ejemplo, el protocolo 2 podría tener una cantidad promedio de nodos vecinos $7$
(contexto) y para esa cantidad de nodos vecinos habría un nodo que ha encontrado
que tiene una tasa de entrega de $0.1$ (métrica), por lo tanto esa es la
información que se dispone para tomar una decisión de que protocolo es mejor en
ese contexto.


Debido a que la posibilidad de que un nodo tenga exactamente el mismo contexto
que otro es baja, es necesario realizar interpolaciones para calcular valores
que se encuentren entre los puntos. Un alternativa es la de tratar de interpolar
un polinomio de grado igual a la cantidad de puntos existentes, pero la ventaja
de precisión que se lograría con esto no justificaría el costo computacional que
requiere el cálculo de las constantes del polinomio, dado que se esta diseñando
el protocolo para que funcione sobre dispositivos móviles que pueden tener bajo
poder de cómputo. Como alternativa, se va a realizar una interpolación
utilizando solo rectas, es decir, se toman los puntos que incluyen al contexto
actual y se genera una recta entre ellos como se muestra en la
\ref{fig:contextos-grafico} en el caso del protocolo 1 donde se muestran
las rectas de interpolación que permiten obtener valores intermedios.

Las rectas consisten en dos valores, $m$ y $b$ tal que al reemplazar esos valores
en la ecuación $y = mx + b$, la recta que se forma pase por los puntos deseados,
que son los que contienen al contexto que se quiere evaluar.


Este proceso de interpolación es realizado para todos los protocolos en todos
los contextos y para todas las métricas. En el caso del protocolo implementado,
estos serían $5$ protocolos, $2$ contextos y $2$ métricas dando un total de $20$
evaluaciones si es que el nodo que realiza la evaluación tiene información de
evaluaciones de los $5$ protocolos. En este punto cada métrica es agrupada
por contexto: hay una lista con las métricas obtenidas en el contexto del
encuentro actual y otra en el contexto del promedio de contacto actual.


Estas evaluaciones solamente indican el desempeño de los protocolos, pero no es
posible tomar una decisión con solo esta información debido a que el \overhead{}
y la tasa de entrega se miden en unidades distintas y además una es mejor cuando
es baja (\overhead) y la otra cuando es alta (tasa de entrega).



\subseccion{Evaluación de la función objetivo}

Para poder utilizar correctamente la ecuación (\ref{eq:normalizar}) es necesario
normalizar la cantidad de copias entre $0$ y $1$ utilizando la ecuación
(\ref{eq:normalizar}), donde $O$ es el \overhead{} estimado o la cantidad de
copias en el tiempo de un protocolo, $mayor$ es el valor máximo que puede tomar
$O$ obtenido mediante simulaciones y finalmente $ON$ es el valor normalizado que
puede utilizarse como entrada de la ecuación (\ref{eq:nota}). En caso de no
existir simulaciones disponibles, se puede aproximar el valor de $mayor$,
incluso sobrestimando su valor real debido a que lo importante es que este valor
sea igual dentro de todos los nodos y no su valor absoluto. En caso que el valor
real supere el valor estimado de $mayor$ se puede tomar la decisión de asignar
al protocolo un valor de la función objetivo de $0$ debido a que aporta con
muchas copias de mensajes a la red.

\begin{equation}
  ON = \frac{O}{mayor}
  \label{eq:normalizar}
\end{equation}


A cada protocolo del estado del arte se le evalua a través de su función
objetivo calculada localmente en cada uno de los nodos de acuerdo a la
información que se ha logrado recolectar de la red en forma de contextos y
métricas de manera de poder tomar la decisión de que protocolo es el más
adecuado para ejecutar en el contexto actual del nodo. Este proceso de
evaluación de la función objetivo debe ser simple de calcular dado que va a ser
ejecutado en un dispositivo móvil donde los recursos son limitados,
especialmente la energía.  Debido a esto, se propone la función de la ecuación
(\ref{eq:nota}) donde se relacionan los valores de la tasa de entrega ($T$) y la
cantidad de copias en el tiempo creadas por el protocolo ($C$), siendo $T$ y $C$
valores entre $0$ y $1$, luego de normalizar $C$.

\newcommand{\graficoEcuacion}[1]{
\pgfplotsset{cycle list/Dark2}
\renewcommand{\colorscheme}{Dark2}

\begin{tikzpicture}
\begin{axis}[
    xlabel={Tasa de entrega},
    ylabel={Copias de mensajes},
    zlabel={Nota},
    xmin=0, xmax=1,
    ymin=0, ymax=1,
    legend style = {
      legend pos = outer north east,
    },
    cycle list name=Dark2,
    width=0.7\textwidth
]

\addplot3+[
  cycle list name=Dark2,
  mesh,
  scatter,
  samples = 10,
  domain=0:1
  ]
  {#1*x + (1 - #1)*(1 - y)};
 
\end{axis}
\end{tikzpicture}
}



\begin{equation}
  f(T, C) = \alpha*T + (1 - \alpha)*(1 - C)
  \label{eq:nota}
\end{equation}

Dado que se quiere que un protocolo con una alta tasa de entrega y baja cantidad
de copias sea el seleccionado, la ecuación debería entregar su valor máximo en
ese caso, en caso contrario se le debería asignar un valor mínimo al protocolo.
Adicionalmente, se incluye un parámetro adicional identificado por la letra
griega $\alpha \in [0, 1]$ que permite variar la importancia que se le da a las
métricas. 



\figura{Valores de la ecuación (\ref{eq:nota}) para $\alpha = 0$.}
{\graficoEcuacion{0}}{fig:valor-0-ecuacion}

\figura{Valores de la ecuación (\ref{eq:nota}) para $\alpha = 0.5$.}
{\graficoEcuacion{0.5}}{fig:valor-0.5-ecuacion}

\figura{Valores de la ecuación (\ref{eq:nota}) para $\alpha = 1$.}
{\graficoEcuacion{1.0}}{fig:valor-1-ecuacion}

En la \ref{fig:valor-0-ecuacion}, \ref{fig:valor-0.5-ecuacion} y
\ref{fig:valor-1-ecuacion} se muestra como varía la ecuación (\ref{eq:nota}) de
la función objetivo para tres valores de $\alpha$. Se puede ver que para $\alpha
= 0$ se le evalúa con un valor alto a protocolos con baja cantidad de copias sin
importar la tasa de entrega, mientras que para un valor de $\alpha = 1$
solamente importa la tasa de entrega. En el caso de $\alpha = 0.5$, se puede
como varía frente a cambios tanto de tasa de entrega como de copias de mensajes.





\subseccion{Cambio de protocolo y del estado}

La programación orientada a objetos tiene como objetivo la encapsulación del
estado del programa en objetos para que sea más fácil razonar acerca del
funcionamiento de este. En el caso de la implementación de un protocolo para el
simulador \theone, estos son programados mediante la creación de una clase que
contiene toda la información del nodo correspondiente con la red.


Cuando se requiere hacer un cambio de un protocolo a otro es necesario lidiar
con el estado interno del protocolo para evitar pérdidas de mensajes o de
métricas. En total hay 12 variables que deben ser copiadas del protocolo
anterior al seleccionado, entre ellas la más importante es \textit{messages},
miembro de la clase \textit{ActiveRouter}. Una vez copiadas estas 12 variables
se puede cambiar el protocolo.

En el caso que el protocolo implemente métricas, como las tablas de probabilidad
de encuentros de \maxprop, \prophet{} y \syf, simplemente se guarda la instancia
del protocolo anterior sin copiar esa información dado que no es de interés para
el nuevo protocolo, pero si es relevante si es que se vuelve a seleccionar el
protocolo anterior.

El protocolo seleccionado a ser ejecutado a continuación es copiado en una
variable dentro del protocolo dinámico el cual redirige las llamadas a los
métodos del protocolo al seleccionado para permitir ejecutar cualquier protocolo
implementado para \theone.

Protocolos como \maxprop, \syf{} y \prophet{} requieren de tratamiento especial
debido a que necesitan ser ejecutados parcialmente en todo momento para
construir sus tablas de probabilidades. Cada vez que un nodo se encuentra con un
vecino, además de llamar al método correspondiente del protocolo actual, se
llama además a los métodos de \maxprop, \syf{} y \prophet{} para que puedan
actualizar sus tablas y puedan utilizar las métricas de probabilidad
inmediatamente cuando son seleccionados y no tengan que pasar por una fase de
espera hasta que puedan conocer la red.



\subseccion{Parámetros internos del protocolo}

A continuación se presenta una lista de los parámetros utilizados para
configurar el protocolo dinámico.

\begin{itemize}
  \item \textbf{$\alpha$}: Toma valores entre $[0, 1]$ para darle prioridad a la tasa de
    entrega o a la cantidad de copias realizadas por un protocolo. Un valor
    cercano a cero indica que para el protocolo dinámico es más importante la
    tasa de entrega y un valor cercano a uno hace que sea más importante la
    cantidad de copias creadas el mensaje.
  \item \textbf{Máximo de copias de mensajes en el tiempo}: Este parámetro se
    utiliza para normalizar la cantidad de copias de mensajes para que se pueda
    utilizar la función objetivo. Se mide en cantidad de mensajes en el tiempo
    ($\frac{mensajes}{tiempo}$).
  \item \textbf{Tiempo de la ventana de métricas}: Definida en segundos, se
    utiliza para limitar la cantidad de información y, por lo tanto, uso de
    memoria en el nodo. Además, cumple la función de descartar la información
    más antigua que puede no ser válida en todos los instantes de la red.
  \item \textbf{Tiempo de la ventana de contextos}: Al igual que la ventana de
    métricas, la ventana de contextos se define en segundos y se utiliza para
    evitar tener información antigua en el cálculo de un contexto. En caso de un
    cambio de movimiento, esta ventana permite recalcular el nuevo contexto al
    descartar información pasada.
  \item \textbf{Tiempo de cambio de protocolo}: El tiempo mínimo que debe
    esperar un nodo antes de cambiar el protocolo actual por uno distinto. Esto
    evita la pérdida de mensajes (en caso de ser un tiempo bajo) y reduce la
    cantidad de cálculos que debe hacer el nodo en cada instante de tiempo. Se
    mide en segundos y su valor máximo es el tiempo de simulación, es decir,
    durante toda la simulación no hay cambios de protocolos.
\end{itemize}







\seccion{Experimentos}

Antes de comenzar a comparar el protocolo dinámico con otros del estado del arte
es necesario encontrar los parámetros adecuados para poder realizar una
comparación con la versión del protocolo dinámico que reduzca el \overhead{} de
mensajes. Existen cuatro parámetros que deben ser encontrados de manera tal de
poder lograr que el protocolo dinámico entregue el mejor desempeño en cuanto a
la tasa de entrega y consumo de energía. Estos parámetros son: tiempo de cambio
de protocolo, tamaño de la ventana de contextos y tamaño de la ventana de las
métricas, todos ellos utilizando la misma unidad de medida, segundos. Además hay
otros dos parámetros que afectan el desempeño del protocolo: valor máximo tasa
de entrega y valor máximo \overhead.

Inicialmente los parámetros se encuentran configurados de acuerdo a lo que dicta
la intuición, por ejemplo, un modelo de movilidad con cambios de movimientos
cortos tendría que tener una ventana de contextos lo suficientemente pequeña
como para poder identificar esos cambios e inmediatamente descartar la historia
pasada, de igual forma el tiempo de cambio de protocolo debería ser del orden de
esta ventana para permitir seleccionar inmediatamente un protocolo que se adapte
mejor a la situación del momento en el nodo.

Los experimentos de los parámetros se basan en los anteriores, es decir, cuando
se encuentra el parámetro que entrega el mejor desempeño de acuerdo a tasa de
entrega y \overhead, este es utilizado en los experimentos posteriores.

Una vez encontrados los parámetros que minimizan el \overhead{} y maximizan la
tasa de entrega, se van a hacer pruebas de escalabilidad del protocolo en cuanto
a tamaño del \textit{buffer}, cantidad de nodos en la red, tamaño de mensajes
y cantidad de mensajes por minuto.

Los modelos de movilidad utilizados son \textit{Post-Disaster Mobiliy Model}
(PDM) \cite{uddin_post-disaster_2009}, \textit{Working Day Mobility Model}
(WDMM) \cite{ekman_working_2008} y  \textit{Valparaíso Mobility Model} (VMM),
este último definido en el \ref{chp:movilidad}. Se utiliza PDM debido a que es
un modelo de movilidad muy utilizado en las publicaciones para probar nuevos
protocolos DTN y con su versión modificada, VMM, permite además probar el
protocolo en un escenario de desastre en una ciudad chilena. En el caso de WDMM,
este modelo de movilidad se utiliza para analizar el uso del protocolo híbrido
dinámico en otras situaciones donde los diferentes movimientos de los nodos de
la red son importantes.


\subseccion{Configuraciones de los experimentos}

Los experimentos para encontrar los parámetros tienen las siguientes
configuraciones:

\tabla{Configuración de las simulaciones para encontrar los parámetros del protocolo dinámico.}{
\pgfplotstabletypeset[
  columns/A/.style={column name=Configuración, string type},
  columns/B/.style={column name=PDM, string type},
  columns/C/.style={column name=VMM, string type},
  columns/D/.style={column name=WDMM, string type},
  every even row/.style={
  before row={\rowcolor[gray]{0.9}}},
  every head row/.style={ before row=\toprule,after row=\midrule},
  every last row/.style={ after row=\bottomrule},
  row sep=\\,
  col sep=&
]{
A & B & C & D\\
Tiempo a simular & 80000s & 80000s & 80000s \\
Nodos totales                         & 330   & 270                   & 160                   \\
Nodos personas                        & 200   & 200                   & 150                   \\
Cambio de protocolo                   & 5000s & 5000s                 & 200s                  \\
Tamaño ventana contextos              & 5000s & 5000s                 & 200s                  \\
Tamaño ventana métricas               & 5000s & 5000s                 & 15000s                \\
Intervalo de generación de mensajes   & 120s  & 120s                  & 120s                   \\
}
}{tbl:parametros-normalizar}
{Elaboración propia, (2017)}


La cantidad de nodos totales y de personas estan basadas en las configuraciones
propuestas en cada una de las publicaciones que definen los respectivos modelos
de movilidad. El tiempo de cambio de protocolo, el tamaño de la ventana de
contextos y el tamaño de la ventana de métricas están basados en simulaciones
realizadas cuando se estaba implementado el protocolo en el simulador,
encontrando que son valores que en un principio entregan un buen
\textit{trade-off} para PDM y VMM. En el caso de WDMM, los valores de $200s$ del
cambio de protocolo y de la ventana de contextos es debido a que se observaron
algunos movimientos en el simulador encontrando que estos ocurrían en pequeños
intervalos de tiempo de $200s$ o menos dentro de las oficinas. El tamaño de la
ventana de métricas para WDMM está basado en las latencias encontradas en la red
debido a que el protocolo debe guardar la información que recibe hasta que
obtenga una nueva desde otro punto de la red. Finalmente, los intervalos de
generación de mensajes se basan en los propuestos por PDM para su escenario de
movilidad.



Además se utilizan las siguientes \textit{seeds} para inicializar los
generadores de números aleatorios: 6339674 como \textit{seed} del modelo de
movilidad, 6477208 como \textit{seed} de la red y 1864168 como \textit{seed} del
generador de mensajes. Por si solos estos números no tienen ningún significado
especial dado que solamente se utilizan como \textit{seed} la generación de
números aleatorios.


Inicialmente a los nodos se les asigna un protocolo de manera
\textit{round-robin} de los 5 del estado del arte mencionados anteriormente:
\maxprop, \syw, \syf, \prophet{} y \epidemic.


El generador de mensajes utilizados en las simulaciones es de todos entre todos,
es decir, cualquier nodo puede generar un mensaje con destino cualquier nodo de
la red. Se utiliza esta forma de generar los mensajes debido a que en una
situación de desastre las personas afectadas necesitan comunicarse con
familiares o autoridades y autoridades con personas.


En cuanto a las pruebas de escalabilidad, se utilizan los siguientes parámetros
luego de realizar los experimentos de la \ref{sec:evaluacion}:



\tabla{Configuración de las simulaciones para pruebas de escalabilidad.}{
\pgfplotstabletypeset[
  columns/A/.style={column name=Configuración, string type},
  columns/B/.style={column name=PDM, string type},
  columns/C/.style={column name=VMM, string type},
  columns/D/.style={column name=WDMM, string type},
  every even row/.style={
  before row={\rowcolor[gray]{0.9}}},
  every head row/.style={ before row=\toprule,after row=\midrule},
  every last row/.style={ after row=\bottomrule},
  row sep=\\,
  col sep=&
]{
A & B & C & D\\
Tiempo a simular & 80000s & 80000s & 80000s \\
Nodos totales                         & 330   & 270                   & 160                   \\
Nodos personas                        & 200   & 200                   & 150                   \\
Cambio de protocolo                   & 7000s & 5000s                 & 5000s                  \\
Tamaño ventana contextos              & 5000s & 10000s                & 1500s                  \\
Tamaño ventana métricas               & 5000s & 25000s                & 5000s                \\
Intervalo de generación de mensajes   & 120s  & 120s                  & 120s                   \\
Valor de normalización del \overhead  & 400   & 400                   & 400                   \\
}
}{tbl:parametros-escalabilidad}
{Elaboración propia, (2017)}


Adicionalmente se utiliza un $\alpha = 0.2$ en PDM, $\alpha = 0.2$ en VMM, y
$\alpha = 0.8$ en WDMM.



Todos los nodos de la red utilizan una interfaz \textit{Bluetooth} 4.0 con un
rango de comunicación de 100 metros y una velocidad de comunicación de 1 Mbps
por segundo. Se utiliza \textit{Bluetooth} en vez de \textit{WiFi} debido a que
está enfocado a comunicar dispositivos móviles con un bajo consumo de energía.


\subseccion{Entorno de ejecución}


Las simulaciones se ejecutaron en un computador con 32 núcleos Xeon E5-2650
corriendo a $2.6$ GHz en un sistema operativo \textit{GNU/Linux Ubuntu 14.04.5
LTS} y \textit{kernel 3.13.0-100-generic} con un total de 125 GB de RAM. En el
\ref{apendice:dinamico} se pueden ver las características del procesador con
mayor detalle en la tabla \ref{tbl:cpuinfo}. 








\seccion{Experimentos de ajuste de parámetros}
\customlabel{sec:evaluacion}{Subcapítulo \arabic{chapter}.\arabic{section}}


\subseccion{Máximos de tasa de entrega y copias de mensajes}


A continuación se presentan los resultados de los experimentos ejecutados para
obtener los valores máximos para normalizar tanto la tasa de entrega como el
\overhead{} en dos escenarios de movilidad: \textit{Post-Disaster Mobility
Model} (PDM), \textit{Working Day Mobility Model} (WDMM) y el modelo creado de
Valparaíso (VMM). Estas simulaciones utilizan $\alpha = 0.5$ dado que es el
punto medio entre la información utilizada para seleccionar el protocolo.



%\pgfplotsset{cycle list/Dark2}
%\pgfplotsset{cycle list/Set1}
\pgfplotsset{cycle list/Set2}
\newcommand{\colorscheme}{Set2}


% Argumentos
% 1: Archivo entrada
% 2: Titulo
% 3: Label x
% 4: Label y
\newcommand{\graficoprotocolos}[4]{
\begin{tikzpicture}
\begin{axis}[
    xlabel={#3},
    ylabel={#4},
    title=#2,
    grid=major,
    legend entries = {Epidemic, MaxProp},
    legend style = {
      legend pos = outer north east,
    },
    mark = *,
    cycle multi list=\colorscheme
  ]
\pgfplotstableread{#1}\mydata;
\addplot table [] {\mydata};
\end{axis}
\end{tikzpicture}
}




\figura{Máximo de la tasa de entrega estimada en el protocolo dinámico en PDM.}
{
\graficoMaximos{Tasa de entrega}
{data/maximos/delivery_PDM_1syw.txt}
{data/maximos/delivery_PDM_1syf.txt}
{data/maximos/delivery_PDM_1epidemic.txt}
{data/maximos/delivery_PDM_1prophet.txt}
{data/maximos/delivery_PDM_1maxprop.txt}
}{fig:tasa-entrega-maximo-pdm}


\figura{Máximo de la tasa de entrega estimada en el protocolo dinámico en WDMM.}
{
\graficoMaximos{Tasa de entrega}
{data/maximos/delivery_wdmm_1syw.txt}
{data/maximos/delivery_wdmm_1syf.txt}
{data/maximos/delivery_wdmm_1epidemic.txt}
{data/maximos/delivery_wdmm_1prophet.txt}
{data/maximos/delivery_wdmm_1maxprop.txt}
}{fig:tasa-entrega-maximo-wdmm}



\figura{Máximo de la tasa de entrega estimada en el protocolo dinámico en VMM.}
{
\graficoMaximos{Tasa de entrega}
{data/maximos/delivery_valpo_1syw.txt}
{data/maximos/delivery_valpo_1syf.txt}
{data/maximos/delivery_valpo_1epidemic.txt}
{data/maximos/delivery_valpo_1prophet.txt}
{data/maximos/delivery_valpo_1maxprop.txt}
}{fig:tasa-entrega-maximo-vmm}

Los gráficos de la \ref{fig:tasa-entrega-maximo-pdm},
\ref{fig:tasa-entrega-maximo-wdmm}, \ref{fig:tasa-entrega-maximo-vmm} muestran
el máximo de todas las tasas de entregas de todos los nodos de la red en un
isntante de tiempo. Debido a la forma de calcular la tasa de entrega siempre va
a tomar valores entre $0$ y $1$, lo que es confirmado por estos experimentos.


\figura{Máximo de la cantidad de copias estimada en el protocolo dinámico en PDM.}
{
\graficoMaximos{Copias de mensajes por hora}
{data/maximos/overhead_PDM_1syw.txt}
{data/maximos/overhead_PDM_1syf.txt}
{data/maximos/overhead_PDM_1epidemic.txt}
{data/maximos/overhead_PDM_1prophet.txt}
{data/maximos/overhead_PDM_1maxprop.txt}
}{fig:copias-maximo-pdm}


\figura{Máximo de la cantidad de copias estimada en el protocolo dinámico en WDMM.}
{
\graficoMaximos{Copias de mensajes por hora}
{data/maximos/overhead_wdmm_1syw.txt}
{data/maximos/overhead_wdmm_1syf.txt}
{data/maximos/overhead_wdmm_1epidemic.txt}
{data/maximos/overhead_wdmm_1prophet.txt}
{data/maximos/overhead_wdmm_1maxprop.txt}
}{fig:copias-maximo-wdmm}



\figura{Máximo de la cantidad de copias estimada en el protocolo dinámico en VMM.}
{
\graficoMaximos{Copias de mensajes por hora}
{data/maximos/overhead_valpo_2syw.txt}
{data/maximos/overhead_valpo_1syf.txt}
{data/maximos/overhead_valpo_1epidemic.txt}
{data/maximos/overhead_valpo_1prophet.txt}
{data/maximos/overhead_valpo_1maxprop.txt}
}{fig:copias-maximo-vmm}

La forma de calcular la cantidad de copias no entrega un valor acotado y como se
puede varía en el tiempo y entre modelo de movilidad. En los modelos de
movilidad PDM y WDMM, se obtiene una cantidad de copias por hora máxima similar
de $350$ en el tiempo mientras que VMM llega hasta los $400$.  Se decide en base
a estos resultados utilizar $400$ como valor máximo de la normalización para los
tres modelos de movilidad. En caso que un protocolo supere en valor máximo de
$400$, se va a considerar como un protocolo con un alto \overhead{} asignándole
un valor de $0$.

%%%%%%%%%%%%%%%%%%%%%%%%%%%%%%%%%%%%%%%%%%%%%%%%%%%%%%%%%%%%%%%%%%%%%%%%%%%%%%%%

\subseccion{Valores de $\alpha$}

A continuación se muestran los resultados obtenidos de las simulaciones para
encontrar que valor de $\alpha$ es el que entrega el mejor desempeño.


%\pgfplotsset{cycle list/Dark2}
%\pgfplotsset{cycle list/Set1}
\pgfplotsset{cycle list/Set2}
\newcommand{\colorscheme}{Set2}


% Argumentos
% 1: Archivo entrada
% 2: Titulo
% 3: Label x
% 4: Label y
\newcommand{\graficoprotocolos}[4]{
\begin{tikzpicture}
\begin{axis}[
    xlabel={#3},
    ylabel={#4},
    title=#2,
    grid=major,
    legend entries = {Epidemic, MaxProp},
    legend style = {
      legend pos = outer north east,
    },
    mark = *,
    cycle multi list=\colorscheme
  ]
\pgfplotstableread{#1}\mydata;
\addplot table [] {\mydata};
\end{axis}
\end{tikzpicture}
}


\figura{Tasa de entrega para distintos valores de $\alpha$ en PDM.}
{
\graficoAlphaDelivery{Tasa de entrega}
{data/alpha/PDM_meta_delivery.txt}
}{fig:tasa-entrega-alpha-pdm}

\figura{\textit{Overhead} para distintos valores de $\alpha$ en PDM.}
{
\graficoAlpha{\# mensajes (\overhead)}
{data/alpha/PDM_meta_overhead.txt}
}{fig:overhead-alpha-pdm}


\figura{Tasa de entrega para distintos valores de $\alpha$ en WDMM.}
{
\graficoAlphaDelivery{Tasa de entrega}
{data/alpha/WDMM_meta_delivery.txt}
}{fig:tasa-entrega-alpha-wdmm}

\figura{\textit{Overhead} para distintos valores de $\alpha$ en WDMM.}
{
\graficoAlpha{\# mensajes (\overhead)}
{data/alpha/WDMM_meta_overhead.txt}
}{fig:overhead-alpha-wdmm}


\figura{Tasa de entrega para distintos valores de $\alpha$ en VMM.}
{
\graficoAlphaDelivery{Tasa de entrega}
{data/alpha/VALPO_meta_delivery.txt}
}{fig:tasa-entrega-alpha-vmm}

\figura{\textit{Overhead} para distintos valores de $\alpha$ en VMM.}
{
\graficoAlpha{\# mensajes (\overhead)}
{data/alpha/VALPO_meta_overhead.txt}
}{fig:overhead-alpha-vmm}


La \ref{fig:tasa-entrega-alpha-pdm} y \ref{fig:overhead-alpha-pdm} muestran los
resultados de las simulaciones para PDM. Se puede ver que a medida que aumenta
el valor de $\alpha$, la tasa de entrega aumenta pero también el \overhead{},
esto es debido a la relación que existe entre ambas métricas: para obtener una
alta tasa de entrega es necesario realizar una mayor cantidad de copias de los
mensajes para aumentar la probabilidad de encontrar a su destino. Para
seleccionar que $\alpha$ es el valor adecuado, una inspección de los resultados
de la tasa de entrega indica que entre $\alpha = 0$ y $\alpha =0.8$ esta se
mantiene cerca de $0.6$ con un aumento en $\alpha = 0.2$. En cuanto al
\overhead, este se dispara a partir de $\alpha = 0.6$ pero alcanza su menor
valor en $0.2$. Utilizando como apoyo una función objetivo $f(T, C) = 0.5*T +
0.5*(1 - C)$, donde se le da igual importancia a ambas métricas, se pueden unir
las dos en un solo gráfico para facilitar su análisis. Esta función se grafica
en la \ref{fig:nota-alpha-pdm} donde el valor máximo es aquel que maximiza la
tasa de entrega y minimiza el \overhead{} en igual medida ($\alpha = 0.5$).
Debido a estos resultados se decide utilizar un valor de $\alpha = 0.2$ para el
resto de las simulaciones en el modelo de movilidad PDM.


En WDMM en la \ref{fig:tasa-entrega-alpha-pdm} y \ref{fig:overhead-alpha-pdm}
tiene una tasa de entrega constante, a pesar de las variaciones en \overhead.
Una causa de estos resultados es la información que puede tener un nodo puede no
ser suficiente debido al bajo \overhead{} obtenido lo que indica que existen
pocos mensajes para transportar las métricas y contextos de la red. Utilizando
la misma función que en PDM, se genera la \ref{fig:nota-alpha-wdmm} donde
finalmente se decide utilizar un $\alpha = 0.8$ que maximiza la tasa de entrega
pero minimiza el \overhead.


Para VMM, existe un aumento de la tasa de entrega en $\alpha = 0.2$ y luego en
$\alpha = 0.9$ y $\alpha = 1$. En los valores altos de $\alpha$ existe un
aumento del \overhead, por lo que se decide utilizar $\alpha = 0.2$. Esto puede
ser corroborado en la \ref{fig:nota-alpha-vmm} donde se grafica la función
objetivo utilizada en el análisis de PDM.




%%%%%%%%%%%%%%%%%%%%%%%%%%%%%%%%%%%%%%%%%%%%%%%%%%%%%%%%%%%%%%%%%%%%%%%%%%%%%%%%

\newpage
\subseccion{Tamaño de la ventana de métricas}


%\pgfplotsset{cycle list/Dark2}
%\pgfplotsset{cycle list/Set1}
\pgfplotsset{cycle list/Set2}
\newcommand{\colorscheme}{Set2}


% Argumentos
% 1: Archivo entrada
% 2: Titulo
% 3: Label x
% 4: Label y
\newcommand{\graficoprotocolos}[4]{
\begin{tikzpicture}
\begin{axis}[
    xlabel={#3},
    ylabel={#4},
    title=#2,
    grid=major,
    legend entries = {Epidemic, MaxProp},
    legend style = {
      legend pos = outer north east,
    },
    mark = *,
    cycle multi list=\colorscheme
  ]
\pgfplotstableread{#1}\mydata;
\addplot table [] {\mydata};
\end{axis}
\end{tikzpicture}
}



\figura{Tasa de entrega para distintos tamaños de ventanas de métricas en PDM.}
{
\graficoTamanoVentanaDelivery{Tasa de entrega}
{data/ventana_historia/PDM_meta_delivery.txt}
}{fig:tasa-entrega-ventana-metricas-pdm}


\figura{\textit{Overhead} para distintos tamaños de ventanas de métricas en PDM.}
{
\graficoTamanoVentana{\# mensajes (\overhead)}
{data/ventana_historia/PDM_meta_overhead.txt}
}{fig:overhead-ventana-metricas-pdm}


\figura{Tasa de entrega para distintos tamaños de ventanas de métricas en WDMM.}
{
\graficoTamanoVentanaDelivery{Tasa de entrega}
{data/ventana_historia/WDMM_meta_delivery.txt}
}{fig:tasa-entrega-ventana-metricas-wdmm}

\figura{\textit{Overhead} para distintos tamaños de ventanas de métricas en WDMM.}
{
\graficoTamanoVentana{\# mensajes (\overhead)}
{data/ventana_historia/WDMM_meta_overhead.txt}
}{fig:overhead-ventana-metricas-wdmm}


\figura{Tasa de entrega para distintos tamaños de ventanas de métricas en VMM.}
{
\graficoTamanoVentanaDelivery{Tasa de entrega}
{data/ventana_historia/VALPO_meta_delivery.txt}
}{fig:tasa-entrega-ventana-metricas-vmm}

\figura{\textit{Overhead} para distintos tamaños de ventanas de métricas en VMM.}
{
\graficoTamanoVentana{\# mensajes (\overhead)}
{data/ventana_historia/VALPO_meta_overhead.txt}
}{fig:overhead-ventana-metricas-vmm}


El tamaño de la ventana de las métricas define que tan antiguo es el historial
que un nodo puede mantener, por lo tanto una gran ventana implica que se tiene
mayor información para tomar decisiones, pero al costo de que esta información
esté desactualizada o un mayor tiempo de ejecución debido a que se deben revisar
más entradas en el protocolo. Además, mientras más información se incluye dentro
del cálculo de los contextos, mayor ruido existe debido a que hay información
pasada que afecta el cálculo del contexto actual.


En el caso de PDM, \ref{fig:tasa-entrega-ventana-metricas-pdm} y
\ref{fig:overhead-ventana-metricas-pdm}, se logra la máxima tasa de entrega
cuando no se tiene ninguna información pero al costo de un alto \overhead. Esto
está relacionado dado que mientras más copias de mensajes hay en la red, mayor
probabilidad de encontrar el destino. Si no se tiene ninguna información del
desempeño de la red entonces no es posible seleccionar ningún protocolo aparte
del que le fue asignado en un principio. A partir de los $10000$ segundos de
ventana, la tasa de entrega se mantiene constante, al igual que el \overhead, lo
que implica que mayor información no va a ayudar a mejorar el desempeño del
protocolo, al menos en este modelo de movilidad. Es por esta razón que para PDM
se selecciona un tamaño de ventana de $5000$ segundos para las siguientes
simulaciones.


Los resultados de WDMM se encuentran en la
\ref{fig:tasa-entrega-ventana-metricas-wdmm} y en la
\ref{fig:overhead-ventana-metricas-wdmm}. En el caso de la tasa de entrega, los
resultados son constantes menos para un valor de $0$ segundos de tamaño de la
ventana de métricas. El gráfico del \overhead{} se corresponde con el de la tasa
de entrega debido a que mayor cantidad de copias ayuda a tener una mayor tasa de
entrega. En el caso de este modelo de movilidad, se utiliza una ventana de
$5000$ segundos debido al bajo \overhead{} obtenido en ese punto.



Para el modelo de movilidad de VMM, existe un comportamiento similar a los
casos anteriores con la diferencia que se puede ver una mejora en la tasa de
entrega a partir de los $25000$ segundos comparado con los tamaños anteriores,
incluso con un \overhead{} relativamente bajo si se compara con los obtenidos en
los tiempos anteriores. Para ayudar al análisis, se pueden ver la
\ref{fig:nota-ventana-metricas-wdmm} donde se unen ambas métricas asignándole un
valor. Se puede ver que el \textit{trade-off} para un $alpha = 0.5$ más alto se
encuentra en $250000$, por lo tanto este es el valor seleccionado para las
próximas simulaciones.


Para PDM y WDMM, se pueden ver el valor de la función objetivo
\ref{fig:nota-ventana-metricas-pdm} y en la \ref{fig:nota-ventana-metricas-wdmm}
que corroboran el tamaño de ventana seleccionado.




%%%%%%%%%%%%%%%%%%%%%%%%%%%%%%%%%%%%%%%%%%%%%%%%%%%%%%%%%%%%%%%%%%%%%%%%%%%%%%%%

\newpage
\subseccion{Tamaño de la ventana de contextos}


%\pgfplotsset{cycle list/Dark2}
%\pgfplotsset{cycle list/Set1}
\pgfplotsset{cycle list/Set2}
\newcommand{\colorscheme}{Set2}


% Argumentos
% 1: Archivo entrada
% 2: Titulo
% 3: Label x
% 4: Label y
\newcommand{\graficoprotocolos}[4]{
\begin{tikzpicture}
\begin{axis}[
    xlabel={#3},
    ylabel={#4},
    title=#2,
    grid=major,
    legend entries = {Epidemic, MaxProp},
    legend style = {
      legend pos = outer north east,
    },
    mark = *,
    cycle multi list=\colorscheme
  ]
\pgfplotstableread{#1}\mydata;
\addplot table [] {\mydata};
\end{axis}
\end{tikzpicture}
}



\figura{Tasa de entrega para distintos tamaños de ventanas de contextos en PDM.}
{
\graficoVentanaContextoDelivery{Tasa de entrega}
{data/ventana_contextos/PDM_meta_delivery.txt}
}{fig:tasa-entrega-ventana-contexto-pdm}


\figura{\textit{Overhead} para distintos tamaños de ventanas de contextos en PDM.}
{
\graficoVentanaContexto{\# mensajes (\overhead)}
{data/ventana_contextos/PDM_meta_overhead.txt}
}{fig:overhead-ventana-contexto-pdm}


\figura{Tasa de entrega para distintos tamaños de ventanas de contextos en WDMM.}
{
\graficoVentanaContextoDelivery{Tasa de entrega}
{data/ventana_contextos/WDMM_meta_delivery.txt}
}{fig:tasa-entrega-ventana-contextos-wdmm}

\figura{\textit{Overhead} para distintos tamaños de ventanas de contextos en WDMM.}
{
\graficoVentanaContexto{\# mensajes (\overhead)}
{data/ventana_contextos/WDMM_meta_overhead.txt}
}{fig:overhead-ventana-contextos-wdmm}


\figura{Tasa de entrega para distintos tamaños de ventanas de contextos en VMM.}
{
\graficoVentanaContextoDelivery{Tasa de entrega}
{data/ventana_contextos/VALPO_meta_delivery.txt}
}{fig:tasa-entrega-ventana-contextos-vmm}

\figura{\textit{Overhead} para distintos tamaños de ventanas de contextos en VMM.}
{
\graficoVentanaContexto{\# mensajes (\overhead)}
{data/ventana_contextos/VALPO_meta_overhead.txt}
}{fig:overhead-ventana-contextos-vmm}




El tamaño de la ventana de contextos limita la cantidad de información que se
utiliza para calcular los contextos de los nodos. Por ejemplo, cuando un nodo
quiere saber su densidad de nodos, utiliza los encuentros pasados para realizar
el cálculo, por lo tanto utilizar un tamaño pequeño permite tener contextos
actualizados con los últimos cambios de movimientos.


Para PDM, la \ref{fig:tasa-entrega-ventana-contexto-pdm} y la
\ref{fig:overhead-ventana-contexto-pdm} muestran los resultados de tasa de
entrega y \overhead{} para varios tamaños de ventanas de contextos. En este
caso, un valor alto de los contextos evita que el protocolo se pueda adaptar a
cambios rápidos de movimiento, por eso la bajada de la tasa de entrega aún
cuando el \overhead{} se mantiene constante. Debido al \overhead{} menor en
$5000$ segundos y a la alta tasa de entrega en ese punto, se escoge un tamaño de
ventana de contextos de $5000$ segundos.


Para el modelo de movilidad WDMM, los resultados son menos concluyentes. La tasa
de entrega es constante para todos los tamaños de ventanas de contextos mientras
que el \overhead{} también podría considerarse constante. Esto quiere decir que
no es tan relevante este parámetro para este modelo de movilidad. Para ayudar a
seleccionar un valor, se utilizaron los valores de la función objetivo de la 
\ref{fig:nota-ventana-contextos-wdmm} siendo $15000$ una nota ligeramente más
grande que el resto.


En VMM la \ref{fig:tasa-entrega-ventana-contextos-vmm} y la
\ref{fig:overhead-ventana-contextos-vmm} se puede ver que la tasa de entrega y
el \overhead{} varían respecto al tamaño de la ventana de contextos, siendo
$10000$ segundos el tamaño que maximiza la tasa de entrega minimizando el
\overhead. Esto se puede ver también en la \ref{fig:nota-ventana-contextos-pdm}
donde se grafican los valores de la función objetivo obtenidos por el protocolo. 





%%%%%%%%%%%%%%%%%%%%%%%%%%%%%%%%%%%%%%%%%%%%%%%%%%%%%%%%%%%%%%%%%%%%%%%%%%%%%%%%

\subseccion{Tiempo de cambio de protocolo}

El tiempo de cambio de protocolo indica por al menos cuanto tiempo un nodo
debe quedarse con su selección de protocolo para evitar problemas de
fluctuaciones. 

Por ejemplo, un nodo podría seleccionar en un tiempo $T$ el protocolo $P_1$,
luego en $T + x$ un protocolo $P_2$ y posteriormente en $T + 2x$ nuevamente
$P_1$, donde $x$ es el tiempo de cambio de protocolo. En caso que este valor sea
muy pequeño, entonces se reduce la oportunidad para que el protocolo $P_2$ pueda
mejorar el desempeño de la red. Además, con un tiempo de cambio de protocolo
pequeño se favorece el uso de óptimos locales debido a que un cambio de
protocolo estaría dado por información particular que haya llegado al nodo en un
espacio de tiempo corto en vez de esperar a tener mayor cantidad de información
de otros sectores de la red.

Una consecuencia de este valor que no tiene relevancia para una posible
implementación del protocolo es el tiempo de ejecución del simulador. Un valor
bajo de tiempo de cambio de protocolo implica mayor cambio y copia de estado de
un protocolo a otro lo que puede provocar diferencias de hasta $17$ horas como
en el caso de VMM. Estos tiempos de ejecución se pueden ver en la
\ref{tbl:tiempos-cambio-pdm}, en la \ref{tbl:tiempos-cambio-wdmm} y en la
\ref{tbl:tiempos-cambio-vmm} para los tres modelos de movilidad.


En este caso, en vez de escoger el valor del tiempo de cambio de protocolo que
minimice el \overhead{} y maximice la tasa de entrega en igual medida
($0.5*entrega + 0.5*overhead$), se van a escoger valores que tengan mayor
\overhead{} y mayor tasa de entrega de manera que el protocolo sea más
competitivo al ser comparado con protocolos del estado del arte. Otra
motivación para escoger valores que no necesariamente minimizan el \overhead{} de
mensajes es la reducción del tiempo de simulación.

A continuación, se presentan los resultados experimentales obtenidos por varios
valores del tiempo de cambio de protocolo para PDM, WDMM y VMM, respecto a tasa
de entrega y \overhead{} de mensajes. En el caso de PDM y VMM se simuló hasta un
tiempo de cambio de protocolo de $12000$ segundos debido a que presentan una
mayor variabilidad en sus resultados por lo que un mayor rango de la variable
puede entregar mayor información, caso contrario al de WDMM.


%\pgfplotsset{cycle list/Dark2}
%\pgfplotsset{cycle list/Set1}
\pgfplotsset{cycle list/Set2}
\newcommand{\colorscheme}{Set2}


% Argumentos
% 1: Archivo entrada
% 2: Titulo
% 3: Label x
% 4: Label y
\newcommand{\graficoprotocolos}[4]{
\begin{tikzpicture}
\begin{axis}[
    xlabel={#3},
    ylabel={#4},
    title=#2,
    grid=major,
    legend entries = {Epidemic, MaxProp},
    legend style = {
      legend pos = outer north east,
    },
    mark = *,
    cycle multi list=\colorscheme
  ]
\pgfplotstableread{#1}\mydata;
\addplot table [] {\mydata};
\end{axis}
\end{tikzpicture}
}




\figura{Tasa de entrega para distintos tiempos de cambio de protocolo en PDM.}
{
\graficoTiempoCambioDeliverySmooth{Tasa de entrega}
{data/tiempo_cambio_router/PDM_meta_delivery.txt}
{data/tiempo_cambio_router/datos_delivery_smooth_pdm.txt}
}{fig:tasa-entrega-tiempo-cambio-pdm}


\figura{\textit{Overhead} para distintos tiempos de cambio de protocolo en PDM.}
{
\graficoTiempoCambioSmooth{\# mensajes (\overhead)}
{data/tiempo_cambio_router/PDM_meta_overhead.txt}
{data/tiempo_cambio_router/datos_overhead_smooth_pdm.txt}
}{fig:overhead-tiempo-cambio-pdm}

En PDM, la \ref{fig:tasa-entrega-tiempo-cambio-pdm} y la
\ref{fig:overhead-tiempo-cambio-pdm} muestran los resultados de tasa de entrega
y \overhead{} de mensajes. A diferencia de los parámetros anteriores, el tiempo
de cambio de protocolo afecta en gran medida estas métricas: la tasa de entrega
se mueve de $0.56$ hasta un máximo de $0.71$, mientras que el \overhead{} se
mueve de $154$ a $315$ copias de mensajes.

De acuerdo a los valores de la función objetivo en la
\ref{fig:nota-tiempo-cambio-pdm}, se debería escoger un tiempo de cambio de
protocolo de $5000$ dado que es el que minimiza el \overhead{} y maximiza la
tasa de entrega, pero para seleccionar un valor que tenga una tasa de entrega
más alta se puede hacer una analogía con el costo beneficio de una inversión,
donde el beneficio es la tasa de entrega y el costo es el \overhead{} o lo que
se paga para obtener una cierta tasa de entrega.

A la tasa de entrega y \overhead{} se le aplica una función de interpolación y
suavizado implementada en R llamada \textit{smooth.spline} que ayuda a
visualizar la pendiente o razón con la cual crecen estas métricas. En la
\ref{tbl:costo-beneficio-pdm} se muestra la tasa de entrega y \overhead{} de la
función \textit{smooth.spline}.  Esta tabla ayuda a ver en que porcentaje se
está dispuesto gastar más \overhead{} para obener un correspondiente aumento de
beneficio de tasa de entrega, usando como costo inicial \overhead{} más bajo.
Por ejemplo, con un aumento de \overhead{} de $5.6\%$, se logra un aumento de
la tasa de entrega de $1.5\%$. En base a esta tabla se utiliza un costo de
$61.57\%$  debido a que con ese aumento se puede lograr hasta un $17\%$ mejor
tasa de entrega. A partir de $61\%$, el costo de \overhead{} se dispara
existiendo una diferencia de $100$ mensajes con un $72\%$ de aumento respecto al
\overhead{} inicial. Finalmente, se utiliza un tiempo de cambio de protocolo de
$7000$ segundos.

\tabla{Tabla de beneficio/costo del tiempo de cambio de protocolo en PDM.}{
\begin{center}
\pgfplotstabletypeset[
  columns/A/.style={column name=Tasa de entrega (Beneficio)},
  columns/B/.style={column name=\textit{Overhead} (Costo)},
  columns/C/.style={column name=\% $\Delta$ Tasa de entrega},
  columns/D/.style={column name=\% $\Delta$ \textit{Overhead}},
  every even row/.style={
  before row={\rowcolor[gray]{0.9}}},
  every head row/.style={ before row=\toprule,after row=\midrule},
  every last row/.style={ after row=\bottomrule},
  row sep=\\,
  col sep=&
]{
A & B & C & D\\
0.538720 & 147.981908 & 0.000000 & 0.000000 \\
0.547052 & 156.266192 & 1.546721 & 5.598173 \\
0.563718 & 172.835001 & 4.640421 & 16.794683 \\
0.580389 & 189.403647 & 7.734818 & 27.991083 \\
0.597063 & 205.969975 & 10.830033 & 39.185916 \\
0.613741 & 222.531929 & 13.925926 & 50.377794 \\
0.630421 & 239.088027 & 17.022158 & 61.565714 \\
0.647103 & 255.635979 & 20.118725 & 72.748130 \\
0.663787 & 272.174685 & 23.215763 & 83.924298 \\
0.680474 & 288.703317 & 26.313234 & 95.093657 \\
0.697160 & 305.222328 & 29.410614 & 106.256516 \\
0.713844 & 321.734858 & 32.507604 & 117.414995 \\
0.730527 & 338.244454 & 35.604299 & 128.571491 \\
}
\end{center}
}{tbl:costo-beneficio-pdm}
{Elaboración propia, (2017)}






\figura{Tasa de entrega para distintos tiempos de cambio de protocolo en WDMM.}
{
\graficoTiempoCambioDelivery{Tasa de entrega}
{data/tiempo_cambio_router/WDMM_meta_delivery.txt}
}{fig:tasa-entrega-tiempo-cambio-wdmm}

\figura{\textit{Overhead} para distintos tiempos de cambio de protocolo en WDMM.}
{
\graficoTiempoCambio{\# mensajes (\overhead)}
{data/tiempo_cambio_router/WDMM_meta_overhead.txt}
}{fig:overhead-tiempo-cambio-wdmm}


Los resultados de la \ref{fig:tasa-entrega-tiempo-cambio-wdmm} y 
\ref{fig:overhead-tiempo-cambio-wdmm} son menos concluyentes para WDMM. No se
aprecia una variación de la tasa de entrega ni del \overhead{} de mensajes, por
lo tanto se utiliza un tiempo de cambio de protocolo de $5000$ segundos para
mantener a las simulaciones por debajo de una hora de ejecución, como se puede
ver en la \ref{tbl:tiempos-cambio-wdmm}.




\figura{Tasa de entrega para distintos tiempos de cambio de protocolo en VMM.}
{
\graficoTiempoCambioDeliverySmooth{Tasa de entrega}
{data/tiempo_cambio_router/VALPO_meta_delivery.txt}
{data/tiempo_cambio_router/datos_delivery_smooth_vmm.txt}
}{fig:tasa-entrega-tiempo-cambio-vmm}

\figura{\textit{Overhead} para distintos tiempos de cambio de protocolo en VMM.}
{
\graficoTiempoCambioSmooth{\# mensajes (\overhead)}
{data/tiempo_cambio_router/VALPO_meta_overhead.txt}
{data/tiempo_cambio_router/datos_overhead_smooth_vmm.txt}
}{fig:overhead-tiempo-cambio-vmm}

La \ref{fig:tasa-entrega-tiempo-cambio-vmm} y
\ref{fig:overhead-tiempo-cambio-vmm} muestran los resultados de varios valores
de tiempos de cambio de protocolo en el escenario de movilidad VMM. Se puede ver
que existe un alto costo de \overhead{} a medida que aumenta la tasa de entrega
cuando se utiliza un tiempo de cambio de protocolo alto. En este caso la tasa de
entrega toma valores desde los $0.53$ hasta los $0.75$ mientras que el
\overhead{} va desde $93.49$ hasta $605.78$.


El mismo análisis de PDM se realiza con VMM en la \ref{tbl:costo-beneficio-vmm},
donde la diferencia con PDM es que es más costoso en términos de
\overhead{} aumentar la tasa de entrega. Por ejemplo, para lograr un aumento de
$34\%$ en la tasa de entrega hay que utilizar $452\%$ más \overhead{} en VMM,
mientras que en PDM un aumento de $128\%$ basta para lograr una mejora similar.

Utilizando un \overhead{} $294$, es decir, un aumento de $214\%$, se puede
obtener una tasa de entrega de $0.62$. Si bien, este aumento de \overhead{} es
significativamente más alto, en las siguientes simulaciones de comparación con
los protocolos del estado del arte se puede ver que sigue siendo menor respecto
a ellos. Finalmente, se escoge un tiempo de cambio de protocolo de $5000$, que
de acuerdo a la \ref{tbl:tiempos-cambio-vmm}, tiene un tiempo de ejecución de
las simulaciones de $13$ horas contra las $24$ de escoger un tiempo de cambio de
protocolo de $500$ segundos.




\tabla{Tabla de beneficio/costo del tiempo de cambio de protocolo en VMM.}{
\begin{center}
\pgfplotstabletypeset[
  columns/A/.style={column name=Tasa de entrega (Beneficio)},
  columns/B/.style={column name=\textit{Overhead} (Costo)},
  columns/C/.style={column name=\% $\Delta$ Tasa de entrega},
  columns/D/.style={column name=\% $\Delta$ \textit{Overhead}},
  every even row/.style={
  before row={\rowcolor[gray]{0.9}}},
  every head row/.style={ before row=\toprule,after row=\midrule},
  every last row/.style={ after row=\bottomrule},
  row sep=\\,
  col sep=&
]{
A & B & C & D\\
0.530055 & 93.498792 & 0.000000 & 0.000000 \\
0.539657 & 115.837321 & 1.811528 & 23.891783 \\
0.558863 & 160.512988 & 5.434847 & 71.673862 \\
0.578071 & 205.182109 & 9.058701 & 119.448940 \\
0.597281 & 249.835526 & 12.682692 & 167.207222 \\
0.616488 & 294.462381 & 16.306314 & 214.937095 \\
0.635690 & 339.053051 & 19.929068 & 262.628267 \\
0.654887 & 383.599276 & 23.550695 & 310.271905 \\
0.674077 & 428.098710 & 27.171027 & 357.865496 \\
0.693257 & 472.556392 & 30.789599 & 405.414433 \\
0.712428 & 516.982971 & 34.406418 & 452.930105 \\
0.731592 & 561.390484 & 38.021905 & 500.425384 \\
0.750753 & 605.789500 & 41.636765 & 547.911577 \\
}
\end{center}
}{tbl:costo-beneficio-vmm}
{Elaboración propia, (2017)}




La mayor tasa de entrega y \overhead{} cuando se tiene un alto tiempo de cambio
de protocolo se puede explicar debido a que no se le permite al protocolo
dinámico minimizar el \overhead{} de manera agresiva, lo que eleva sus niveles
respecto a tiempos de cambio de protocolo menores. Con un alto \overhead{}
existen más copias de mensajes en la red lo que aumentan la probabilidad de que
un mensaje sea entregado, es decir, aumenta la tasa de entrega.



%%%%%%%%%%%%%%%%%%%%%%%%%%%%%%%%%%%%%%%%%%%%%%%%%%%%%%%%%%%%%%%%%%%%%%%%%%%%%%%%



\seccion{Experimentos para comparar con protocolos del estado del arte}

A continuación se presentan los resultados de los experimentos para comparar la
escalabilidad del protocolo dinámico con los protocolos del estado del arte.

%\pgfplotsset{cycle list/Dark2}
%\pgfplotsset{cycle list/Set1}
\pgfplotsset{cycle list/Set2}
\newcommand{\colorscheme}{Set2}


% Argumentos
% 1: Archivo entrada
% 2: Titulo
% 3: Label x
% 4: Label y
\newcommand{\graficoprotocolos}[4]{
\begin{tikzpicture}
\begin{axis}[
    xlabel={#3},
    ylabel={#4},
    title=#2,
    grid=major,
    legend entries = {Epidemic, MaxProp},
    legend style = {
      legend pos = outer north east,
    },
    mark = *,
    cycle multi list=\colorscheme
  ]
\pgfplotstableread{#1}\mydata;
\addplot table [] {\mydata};
\end{axis}
\end{tikzpicture}
}

\subseccion{Tamaños de mensajes}


Se realizaron experimentos para analizar como el nuevo protocolo dinámico escala
respecto  protocolos del estado del arte con distintos tamaños de mensajes
desde 1 kilobyte, suficiente para enviar mensajes de texto, hasta 4 megabytes,
tamaño de una imagen pequeña.

\newcommand{\mensajes}{distintos tamaños de mensajes}

\figura{Tasa de entrega para \mensajes{} en PDM.}
{
\graficoDeliveryProtocolosTamano
{Tamaño de mensaje (kilobytes)}
{data/tamano_mensaje/PDM_syw_delivery.txt}
{data/tamano_mensaje/PDM_syf_delivery.txt}
{data/tamano_mensaje/PDM_epidemic_delivery.txt}
{data/tamano_mensaje/PDM_prophet_delivery.txt}
{data/tamano_mensaje/PDM_maxprop_delivery.txt}
{data/tamano_mensaje/PDM_meta_delivery.txt}
}{fig:tasa-entrega-tamano-mensajes-pdm}


\figura{\textit{Overhead} de mensajes para \mensajes{} en PDM.}
{
\graficoProtocolosTamano
{\# mensajes (\overhead)}
{Tamaño de mensaje (kilobytes)}
{data/tamano_mensaje/PDM_syw_overhead.txt}
{data/tamano_mensaje/PDM_syf_overhead.txt}
{data/tamano_mensaje/PDM_epidemic_overhead.txt}
{data/tamano_mensaje/PDM_prophet_overhead.txt}
{data/tamano_mensaje/PDM_maxprop_overhead.txt}
{data/tamano_mensaje/PDM_meta_overhead.txt}
}{fig:overhead-tamano-mensaje-pdm}



\figura{Tasa de entrega para \mensajes{} en WDMM.}
{
\graficoDeliveryProtocolosTamano
{Tamaño de mensaje (kilobytes)}
{data/tamano_mensaje/WDMM_syw_delivery.txt}
{data/tamano_mensaje/WDMM_syf_delivery.txt}
{data/tamano_mensaje/WDMM_epidemic_delivery.txt}
{data/tamano_mensaje/WDMM_prophet_delivery.txt}
{data/tamano_mensaje/WDMM_maxprop_delivery.txt}
{data/tamano_mensaje/WDMM_meta_delivery.txt}
}{fig:tasa-entrega-tamano-mensajes-wdmm}


\figura{\textit{Overhead} de mensajes para \mensajes{} en WDMM.}
{
\graficoProtocolosTamano
{\# mensajes (\overhead)}
{Tamaño de mensaje (kilobytes)}
{data/tamano_mensaje/WDMM_syw_overhead.txt}
{data/tamano_mensaje/WDMM_syf_overhead.txt}
{data/tamano_mensaje/WDMM_epidemic_overhead.txt}
{data/tamano_mensaje/WDMM_prophet_overhead.txt}
{data/tamano_mensaje/WDMM_maxprop_overhead.txt}
{data/tamano_mensaje/WDMM_meta_overhead.txt}
}{fig:overhead-tamano-mensaje-wdmm}


\figura{Tasa de entrega para \mensajes{}  en VMM.}
{
\graficoDeliveryProtocolosTamano
{Tamaño de mensaje (kilobytes)}
{data/tamano_mensaje/VALPO_syw_delivery.txt}
{data/tamano_mensaje/VALPO_syf_delivery.txt}
{data/tamano_mensaje/VALPO_epidemic_delivery.txt}
{data/tamano_mensaje/VALPO_prophet_delivery.txt}
{data/tamano_mensaje/VALPO_maxprop_delivery.txt}
{data/tamano_mensaje/VALPO_meta_delivery.txt}
}{fig:tasa-entrega-tamano-mensajes-vmm}


\figura{\textit{Overhead} de mensajes para \mensajes{} en VMM.}
{
\graficoProtocolosTamano
{\# mensajes (\overhead)}
{Tamaño de mensaje (kilobytes)}
{data/tamano_mensaje/VALPO_syw_overhead.txt}
{data/tamano_mensaje/VALPO_syf_overhead.txt}
{data/tamano_mensaje/VALPO_epidemic_overhead.txt}
{data/tamano_mensaje/VALPO_prophet_overhead.txt}
{data/tamano_mensaje/VALPO_maxprop_overhead.txt}
{data/tamano_mensaje/VALPO_meta_overhead.txt}
}{fig:overhead-tamano-mensaje-vmm}


La \ref{fig:tasa-entrega-tamano-mensajes-pdm},
\ref{fig:tasa-entrega-tamano-mensajes-wdmm} y 
\ref{fig:tasa-entrega-tamano-mensajes-vmm} muestran las tasas de entrega de los
protocolos del estado del arte junto con el protocolo dinámico en los modelos de
movilidad PDM, WDMM y VMM. En la \ref{fig:overhead-tamano-mensaje-pdm},
\ref{fig:overhead-tamano-mensaje-wdmm} y \ref{fig:overhead-tamano-mensaje-vmm}
se muestran los resultados del \overhead{} promedio generado por cada uno de los
protocolos para varios tamaños de mensajes. 

Se puede ver que el que tiene el mejor desempeño de todos los protocolos es
\maxprop, protocolo que además no sufre ningún cambio frente a los cambios de
tamaños de mensajes en los modelos de movilidad PDM y VMM. Junto con \syf,
\epidemic{} y \prophet{}, superan en cuanto a tasa de entrega al protocolo
híbrido dinámico debido a que su función es maximizar la tasa de entrega sin
tomar en cuenta el \overhead{} generado, incluso, debido a la relación que
existe entre el \overhead{} y la tasa de entrega, a mayor \overhead{}, mejor
tasa de entrega. En los gráficos de \overhead{} se puede ver que los protocolos
que tienen buena tasa de entrega en PDM y VMM además son los que más generan
copias de mensajes en la red. El protocolo híbrido dinámico, con mensajes de
$1000$ kilobytes o menos, tiene un \overhead{} mayor que \epidemic{} y
\prophet{} en el caso de PDM sin presentar una mejor tasa de entrega que estos
protocolos para estos tamaños de mensajes indicando que el protocolo dinámico,
para PDM, es mejor utilizarlo cuando los mensajes de la red sean de un tamaño
superior a $1000$ kilobytes. Mismo análisis puede ser realizado para el caso de
VMM donde \epidemic{} y \prophet{} para mensajes con un tamaño menor a $1000$
kilobytes tienen mejor desempeño que para el protocolo dinámico.

En el modelo de movilidad WDMM, la \ref{fig:tasa-entrega-tamano-mensajes-wdmm}
y la \ref{fig:overhead-tamano-mensaje-wdmm} muestran que el protocolo híbrido
dinámico es menos competitivo en este escenario de movilidad que en los otros
dos modelos de moviliad. En general, \prophet{} y \syf{} son mejores opciones de
protocolos para este escenario dado que obtienen mejor tasa de entrega con un
menor \overhead. A pesar de esto, \syf{} es el protocolo más lento del modelo de
movilidad como se puede ver en la \ref{fig:latencia-tamano-wdmm}.

Para tamaños de mensajes superiores a $1000$ kilobytes, el protocolo dinámico
continua teniendo una tasa de entrega inferior a \epidemic, \prophet, \syf{} y
\maxprop, pero al mismo tiempo tiene un menor \overhead. Este es el
\textit{trade-off} que se busca con el protocolo híbrido dinámico, reducir el
\overhead{} que está directamente relacionado con el consumo de energía.

Respecto al efecto que tiene el tamaño de mensaje en los protocolos, un tamaño
de mensaje menor permite que un nodo pueda transportar una mayor cantidad de
mensajes y transmitirlos de un nodo a otro más rapidamente que mensajes de
mayor tamaño, razón por la cual existe un aumento de la tasa para mensajes
pequeños.

En la figura \ref{fig:energia-tamano-pdm}, \ref{fig:energia-tamano-WDMM} y
\ref{fig:energia-tamano-vmm} se muestran los resultados del consumo de energía
de los protocolos que está directamente relacionado con el \overhead{} de
mensajes.











\subseccion{Densidad de personas}


Para analizar como afecta la densidad de personas, se realizaron experimentos
con $25\%$, $50\%$, $75\%$ y $100\%$ de las personas participantes en la red
original.





\figura{Tasa de entrega para distintas densidades de nodos en PDM.}
{
\graficoDeliveryProtocolos
{Densidad de personas (\%)}
{data/personas/PDM_syw_delivery.txt}
{data/personas/PDM_syf_delivery.txt}
{data/personas/PDM_epidemic_delivery.txt}
{data/personas/PDM_prophet_delivery.txt}
{data/personas/PDM_maxprop_delivery.txt}
{data/personas/PDM_meta_delivery.txt}
}{fig:tasa-entrega-densidad-pdm}


\figura{\textit{Overhead} de mensajes para distintas densidades de personas en PDM.}
{
\graficoProtocolos
{\# mensajes (\overhead)}
{Densidad de personas (\%)}
{data/personas/PDM_syw_overhead.txt}
{data/personas/PDM_syf_overhead.txt}
{data/personas/PDM_epidemic_overhead.txt}
{data/personas/PDM_prophet_overhead.txt}
{data/personas/PDM_maxprop_overhead.txt}
{data/personas/PDM_meta_overhead.txt}
}{fig:overhead-densidad-pdm}


\figura{Tasa de entrega para distintas densidades de personas en WDMM.}
{
\graficoDeliveryProtocolos
{Densidad de personas (\%)}
{data/personas/WDMM_syw_delivery.txt}
{data/personas/WDMM_syf_delivery.txt}
{data/personas/WDMM_epidemic_delivery.txt}
{data/personas/WDMM_prophet_delivery.txt}
{data/personas/WDMM_maxprop_delivery.txt}
{data/personas/WDMM_meta_delivery.txt}
}{fig:tasa-entrega-densidad-wdmm}


\figura{\textit{Overhead} de mensajes para distintas densidades de personas en WDMM.}
{
\graficoProtocolos
{\# mensajes (\overhead)}
{Densidad de personas (\%)}
{data/personas/WDMM_syw_overhead.txt}
{data/personas/WDMM_syf_overhead.txt}
{data/personas/WDMM_epidemic_overhead.txt}
{data/personas/WDMM_prophet_overhead.txt}
{data/personas/WDMM_maxprop_overhead.txt}
{data/personas/WDMM_meta_overhead.txt}
}{fig:overhead-densidad-wdmm}


\figura{Tasa de entrega para distintas densidades de personas en VMM.}
{
\graficoDeliveryProtocolos
{Densidad de personas (\%)}
{data/personas/VALPO_syw_delivery.txt}
{data/personas/VALPO_syf_delivery.txt}
{data/personas/VALPO_epidemic_delivery.txt}
{data/personas/VALPO_prophet_delivery.txt}
{data/personas/VALPO_maxprop_delivery.txt}
{data/personas/VALPO_meta_delivery.txt}
}{fig:tasa-entrega-densidad-vmm}


\figura{\textit{Overhead} de mensajes para distintas densidades de personas en VMM.}
{
\graficoProtocolos
{\# mensajes (\overhead)}
{Densidad de personas (\%)}
{data/personas/VALPO_syw_overhead.txt}
{data/personas/VALPO_syf_overhead.txt}
{data/personas/VALPO_epidemic_overhead.txt}
{data/personas/VALPO_prophet_overhead.txt}
{data/personas/VALPO_maxprop_overhead.txt}
{data/personas/VALPO_meta_overhead.txt}
}{fig:overhead-densidad-vmm}


Para PDM, la \ref{fig:tasa-entrega-densidad-pdm} y
\ref{fig:overhead-densidad-pdm} muestran los resultados de las simulaciones para
distintos porcentajes de densidades de personas. Se puede ver que el protocolo
híbrido dinámico obtiene una buena tasa de entrega, y bajo \overhead{} para
todas las densidades obteniendo un buen \textit{trade-off} respecto a los
protocolos del estado del arte exceptuando el caso de \syf{} donde con un  20\%
de personas se obtiene mejor tasa de entrega con un mismo \overhead, lo que
quiere decir que la reducción de consumo de energía que realiza el protocolo
híbrido dinámico perjudica en mayor medida a la tasa de entrega que la realizada
por \syf. En el resto de las densidades la mejora de \overhead{} si es
significativa respecto al resto de los protocolos.  Estos resultados indican que
para PDM el protocolo quiere más de 50 (25\%) de personas en la simulación o 5
personas por vecindario para funcionar de manera correcta.

Para el caso de WDMM, la \ref{fig:tasa-entrega-densidad-wdmm} y la
\ref{fig:overhead-densidad-wdmm} muestran los resultados de tasa de entrega y
\overhead, pero nuevamente el protocolo dinámico es menos competitivo en WDMM
con una tasa de entrega menor que \maxprop{} y \syf{}. Esto se puede explicar
debido a la falta de información que tiene el protocolo para tomar la decisión
de que protocolo seleccionar y al estrecho margén de mejora: en PDM y VMM,
existen diferencias entre las tasas de entrega de los protocolos, mientras que
en WDMM no hay mucha distancia en el desempeño de tasa de entrega. Para bajas
densidades de personas se logra superar al resto de los protocolos y equiparar
el \overhead{} de \prophet{}, pero sin lograr mejorar la tasa de entrega lo que
indica que el protocolo dinámico funciona en pequeñas redes. En este escenario
de movilidad nuevamente \syf{} es el protocolo más lento de acuerdo a la
\ref{fig:latencia-densidad-vmm} lo que dependiendo de la aplicación que se le
quiera dar a la red puede hacer de este protocolo no adecuado si es que se
necesitan bajas latencias.


El modelo de movilidad VMM, al ser una variación de PDM, se obtienen resultados
parecidos en la \ref{fig:tasa-entrega-densidad-vmm} y la
\ref{fig:overhead-densidad-vmm}. El protocolo dinámico un dempeño menor en cuato
a tasa de entrega pero con un \overhead{} menor, logrando un buen
\textit{trade-off} entre consumo de energía y tasa de entrega.


La \ref{fig:energia-personas-pdm}, \ref{fig:energia-personas-WDMM} y
\ref{fig:energia-personas-vmm} muestran los resultados de consumo de energía
que se relacionan con los gráficos de \overhead.




\subseccion{Generación de mensajes}

Para medir como reaccionan los protocolos frente a redes con un alto flujo de
mensajes, se varía la cantidad de tiempo que pasa entre mensaje generado. Por
ejemplo, un tiempo de generación de mensajes de 30 quiere decir que cada 30
segundos se genera un mensaje.



\figura{Tasa de entrega para distintos tiempos de generación de mensajes en PDM.}
{
\graficoDeliveryProtocolos
{Segundos por mensaje generado}
{data/generacion_mensajes/PDM_syw_delivery.txt}
{data/generacion_mensajes/PDM_syf_delivery.txt}
{data/generacion_mensajes/PDM_epidemic_delivery.txt}
{data/generacion_mensajes/PDM_prophet_delivery.txt}
{data/generacion_mensajes/PDM_maxprop_delivery.txt}
{data/generacion_mensajes/PDM_meta_delivery.txt}
}{fig:tasa-entrega-generacion-pdm}

\figura{\textit{Overhead} de mensajes para distintos tiempos de generación de
mensajes en PDM.}
{
\graficoProtocolos
{\# mensajes (\overhead)}
{Segundos por mensaje generado}
{data/generacion_mensajes/PDM_syw_overhead.txt}
{data/generacion_mensajes/PDM_syf_overhead.txt}
{data/generacion_mensajes/PDM_epidemic_overhead.txt}
{data/generacion_mensajes/PDM_prophet_overhead.txt}
{data/generacion_mensajes/PDM_maxprop_overhead.txt}
{data/generacion_mensajes/PDM_meta_overhead.txt}
}{fig:overhead-generacion-pdm}


\figura{Tasa de entrega para distintos tiempos de generación de mensajes en WDMM.}
{
\graficoDeliveryProtocolos
{Segundos por mensaje generado}
{data/generacion_mensajes/WDMM_syw_delivery.txt}
{data/generacion_mensajes/WDMM_syf_delivery.txt}
{data/generacion_mensajes/WDMM_epidemic_delivery.txt}
{data/generacion_mensajes/WDMM_prophet_delivery.txt}
{data/generacion_mensajes/WDMM_maxprop_delivery.txt}
{data/generacion_mensajes/WDMM_meta_delivery.txt}
}{fig:tasa-entrega-generacion-wdmm}

\figura{\textit{Overhead} de mensajes para distintos tiempos de generación de
mensajes en WDMM.}
{
\graficoProtocolos
{\# mensajes (\overhead)}
{Segundos por mensaje generado}
{data/generacion_mensajes/WDMM_syw_overhead.txt}
{data/generacion_mensajes/WDMM_syf_overhead.txt}
{data/generacion_mensajes/WDMM_epidemic_overhead.txt}
{data/generacion_mensajes/WDMM_prophet_overhead.txt}
{data/generacion_mensajes/WDMM_maxprop_overhead.txt}
{data/generacion_mensajes/WDMM_meta_overhead.txt}
}{fig:overhead-generacion-wdmm}

\figura{Tasa de entrega para distintos tiempos de generación de mensajes en VMM.}
{
\graficoDeliveryProtocolos
{Segundos por mensaje generado}
{data/generacion_mensajes/VALPO_syw_delivery.txt}
{data/generacion_mensajes/VALPO_syf_delivery.txt}
{data/generacion_mensajes/VALPO_epidemic_delivery.txt}
{data/generacion_mensajes/VALPO_prophet_delivery.txt}
{data/generacion_mensajes/VALPO_maxprop_delivery.txt}
{data/generacion_mensajes/VALPO_meta_delivery.txt}
}{fig:tasa-entrega-generacion-vmm}

\figura{\textit{Overhead} de mensajes para distintos tiempos de generación de
mensajes en VMM.}
{
\graficoProtocolos
{\# mensajes (\overhead)}
{Segundos por mensaje generado}
{data/generacion_mensajes/VALPO_syw_overhead.txt}
{data/generacion_mensajes/VALPO_syf_overhead.txt}
{data/generacion_mensajes/VALPO_epidemic_overhead.txt}
{data/generacion_mensajes/VALPO_prophet_overhead.txt}
{data/generacion_mensajes/VALPO_maxprop_overhead.txt}
{data/generacion_mensajes/VALPO_meta_overhead.txt}
}{fig:overhead-generacion-vmm}


En general los tres modelos de movilidad presentan las mismas curvas en los
gráficos. Esto es debido a que al existir una mayor cantidad de mensajes, existe
una mayor competencia por los recursos de la red, tanto de ancho de banda de
comunicación como espacio dentro de los \textit{buffers}.


En PDM, \ref{fig:tasa-entrega-generacion-pdm} y
\ref{fig:overhead-generacion-pdm}, se puede ver como a medida que más mensajes
existen en la red, menor es la tasa de entrega causando que el \overhead{}
también baje debido a que su valor es calculado respecto a los mensajes
entregados a los destinos. A pesar que el \overhead{} disminuye, existe un
aumento del consumo de energía debido a que aumenta la cantidad de mensajes que
son transmitidos entre nodos como se puede ver en la
\ref{fig:energia-generacion-pdm}, mostrando que a pesar de la relación que
existe entre \overhead{} y consumo de energía, la dependencia del \overhead{} de
la tasa de entrega, como se muestra en la ecuación (\ref{eq:overhead}), puede
llevar a conclusiones equivocadas y es necesario presentar ambos gráficos
juntos, especialmente cuando la curva del \overhead{} y de la energía tienen
forma diferente.

En términos de consumo de energía y tasa de entrega, se logra un
\textit{trade-off} significativo entre $30$ segundos por mensaje y $120$
segundos por mensaje logrando tener un consumo menor que los protocolos del
estado del arte en ese rango. En el caso de $1$ segundos por mensaje, todos los
nodos tienen una baja tasa de entrega debido a la saturación de los
\textit{buffers} causando que muchos mensajes sean descartados para hacer
espacio a mensajes nuevos, como se puede ver en la
\ref{fig:descartados-generacion-pdm}.

En el caso de WDMM, la \ref{fig:tasa-entrega-generacion-wdmm} y la
\ref{fig:overhead-generacion-wdmm} muestran que nuevamente el protocolo híbrido
dinámico tiene un peor resultado que \syf{} en tasa de entrega y en \overhead,
pero tiene una alta lantencia de acuerdo a la
\ref{fig:latencia-generacion-wdmm}.  Si se mira el gráfico del consumo de
energía en la \ref{fig:energia-generacion-WDMM} se puede ver que el consumo de
energía se dispara cuando hay muchos mensajes en la red, ocurriendo el mismo
fenómeno que en PDM donde los mensajes son descartados por la saturación de los
\textit{buffers} de los protocolos, como se puede ver en la
\ref{fig:descartados-generacion-WDMM}.



En VMM los resultados son similares a los de PDM. La
\ref{fig:tasa-entrega-generacion-vmm} y \ref{fig:overhead-generacion-vmm}
muestra que el protocolo dinámico presenta una ventaja en cuanto al consumo de
energía y se obtienen tasas de entrega cercanas a las de los protocolos del
estado del arte. Nuevamente, la razón de la baja tasa de entrega cuando se
genera un mensaje por segundo es debido a la alta cantidad de mensajes que son
descartados en los \textit{buffers} para hacer espacio a nuevos mensajes
\ref{fig:descartados-generacion-vmm}. En la \ref{fig:energia-generacion-vmm} se
pueden ver los resultados de la energía donde se muestra el aumento de cuando
una alta cantidad de mensajes circulan en la red, causado por la necesidad de
retransmitir mensajes descartados.



\subseccion{Tamaños de los \textit{buffers}}

Para analizar que tan bien administra un protocolo el \textit{buffer} de los
dispositivos sobre los cual se está ejecutando, es que se realizan pruebas con
distintos tamaños de \textit{buffer}.


\newcommand{\buffers}{distintos tamaños de \textit{buffer}}

\figura{Tasa de entrega para \buffers{} en PDM.}
{
\graficoDeliveryProtocolos
{Tamaño del \textit{buffer} (megabytes)}
{data/buffer/PDM_syw_delivery.txt}
{data/buffer/PDM_syf_delivery.txt}
{data/buffer/PDM_epidemic_delivery.txt}
{data/buffer/PDM_prophet_delivery.txt}
{data/buffer/PDM_maxprop_delivery.txt}
{data/buffer/PDM_meta_delivery.txt}
}{fig:tasa-entrega-buffer-pdm}

\figura{\textit{Overhead} de mensajes para \buffers{} en PDM.}
{
\graficoProtocolos
{\# mensajes (\overhead)}
{Tamaño del \textit{buffer} (megabytes)}
{data/buffer/PDM_syw_overhead.txt}
{data/buffer/PDM_syf_overhead.txt}
{data/buffer/PDM_epidemic_overhead.txt}
{data/buffer/PDM_prophet_overhead.txt}
{data/buffer/PDM_maxprop_overhead.txt}
{data/buffer/PDM_meta_overhead.txt}
}{fig:overhead-buffer-pdm}


\figura{Tasa de entrega para \buffers{} en WDMM.}
{
\graficoDeliveryProtocolos
{Tamaño del \textit{buffer} (megabytes)}
{data/buffer/WDMM_syw_delivery.txt}
{data/buffer/WDMM_syf_delivery.txt}
{data/buffer/WDMM_epidemic_delivery.txt}
{data/buffer/WDMM_prophet_delivery.txt}
{data/buffer/WDMM_maxprop_delivery.txt}
{data/buffer/WDMM_meta_delivery.txt}
}{fig:tasa-entrega-buffer-wdmm}

\figura{\textit{Overhead} de mensajes para \buffers{} en WDMM.}
{
\graficoProtocolos
{\# mensajes (\overhead)}
{Tamaño del \textit{buffer} (megabytes)}
{data/buffer/WDMM_syw_overhead.txt}
{data/buffer/WDMM_syf_overhead.txt}
{data/buffer/WDMM_epidemic_overhead.txt}
{data/buffer/WDMM_prophet_overhead.txt}
{data/buffer/WDMM_maxprop_overhead.txt}
{data/buffer/WDMM_meta_overhead.txt}
}{fig:overhead-buffer-wdmm}

\figura{Tasa de entrega para \buffers{} en VMM.}
{
\graficoDeliveryProtocolos
{Tamaño del \textit{buffer} (megabytes)}
{data/buffer/VALPO_syw_delivery.txt}
{data/buffer/VALPO_syf_delivery.txt}
{data/buffer/VALPO_epidemic_delivery.txt}
{data/buffer/VALPO_prophet_delivery.txt}
{data/buffer/VALPO_maxprop_delivery.txt}
{data/buffer/VALPO_meta_delivery.txt}
}{fig:tasa-entrega-buffer-vmm}

\figura{\textit{Overhead} de mensajes para \buffers{} en VMM.}
{
\graficoProtocolos
{\# mensajes (\overhead)}
{Tamaño del \textit{buffer} (megabytes)}
{data/buffer/VALPO_syw_overhead.txt}
{data/buffer/VALPO_syf_overhead.txt}
{data/buffer/VALPO_epidemic_overhead.txt}
{data/buffer/VALPO_prophet_overhead.txt}
{data/buffer/VALPO_maxprop_overhead.txt}
{data/buffer/VALPO_meta_overhead.txt}
}{fig:overhead-buffer-vmm}


La \ref{fig:tasa-entrega-buffer-pdm} y 
\ref{fig:overhead-buffer-pdm} muestran los resultados de los protocolos en el
modelo de movilidad PDM. Se puede ver que \maxprop{} mantiene una tasa de
entrega constante al igual que \syw, mientras los otros protocolos, incluyendo
el híbrido dinámico, sufren una caída de su tasa de entrega. En cuando al
protocolo híbrido dinámico, su comportamiento es similar al de los demás
protocolos del estado del arte mostrando una caída de la tasa de entrega para
\textit{buffers} menores a $128$ megabytes. A pesar de esto, la cantidad de
copias de mensajes y el consumo de energía, como se puede ver en la
\ref{fig:energia-buffer-pdm}, es superior para todos los protocolos menos para
\syw, el protocolo que menos cantidad de copias genera. La baja tasa de entrega
en los protocolos se debe a que con un \textit{buffer} pequeño es necesario
descartar mensaje para hacer espacio a nuevo, causando que los mensajes se
pierdan. Esta es la misma razón por la cual hay un mayor consumo de energía con
un \textit{buffer} pequeño, dado que se necesita volver a retransmitir aquellos
mensajes que fueron descartados. En la \ref{fig:energia-buffer-pdm} se puede ver
el consumo de energía los nodos donde se ve el efecto que tiene la retransmisión
de mensajes en esta métrica.


En el caso de WDMM, en la \ref{fig:tasa-entrega-buffer-wdmm} y
\ref{fig:overhead-buffer-wdmm} se puede ver que todos los protocolos son
afectados por el uso de un \textit{buffer} más pequeño. En el caso de \syf, este
protocolo obtiene un mejor \textit{trade-off} respecto al costo de obtener una
buena tasa de entrega al utilizar más copias de mensajes. A pesar de esto,
\syf{} es el protocolo con mayor latencia de acuerdo a la
\ref{fig:latencia-buffer-pdm}.

Para VMM, la \ref{fig:tasa-entrega-buffer-vmm} y la
\ref{fig:overhead-buffer-vmm} muestran que para un \textit{buffer} de 64
megabytes, el protocolo híbrido dinámico tiene mejor tasa de entrega y mejor
\overhead{} que \prophet{} y \epidemic. Respecto a \maxprop{} y \syf, ambos
protocolos tienen un \overhead{} a pesar de tener una alta tasa de entrega. En
la \ref{fig:energia-buffer-vmm} se puede ver que el protocolo híbrido dinámico
logra reducir significativamente el consumo de energía en la red, cambiando
\overhead{} por tasa de entrega.


En el caso de \syw, en todas las simulaciones de tamaños de \textit{buffer}
mantiene un desempeño constante. Esto se explica por la poca cantidad de copias
de mensajes que genera debido a su limitación de la inundación de la red.



\subseccion{Análisis de los resultados de WDMM}

En WDMM, el protocolo dinámico obtiene un desempeño menor que en PDM y VMM,
donde algunos protocolos del estado del arte tienen mejor tasa de entrega y un
\overhead{} de mensajes más bajo que el protocolo dinámico. Analizando la
\ref{fig:overhead-generacion-wdmm} se puede ver que el protocolo híbrido
dinámico obtiene un \overhead{} o inferior que los protocolos del estado del
arte, exceptuando \syw, lo que quiere decir que en ese punto el protocolo
funciona correctamente logrando seleccionar un protocolo del estado del arte
para reducir la cantidad de copias de mensajes.

Para poder encontrar la razón de los resultados, hay que comparar las
diferencias entre el protocolo que se selecciona cuando cada 1 segundo se genera
un mensaje, cuando cada 30 segundos se genera un mensaje y cuando cada 120
segundos se genera un mensaje.


\figura{Protocolo del estado del arte seleccionados por el protocolo híbrido
dinámico en el tiempo de simulación en WDMM para 1 segundo por mensaje generado.}
{
\graficoProtocolosTiempo
{\% de nodos que han seleccionado el protocolo}
{Tiempo de simulación (segundos)}
{data/protocolos/PERSONA_syw_routers_true_nodos_wdmm1.txt}
{data/protocolos/PERSONA_syf_routers_true_nodos_wdmm1.txt}
{data/protocolos/PERSONA_epidemic_routers_true_nodos_wdmm1.txt}
{data/protocolos/PERSONA_prophet_routers_true_nodos_wdmm1.txt}
{data/protocolos/PERSONA_maxprop_routers_true_nodos_wdmm1.txt}
}{fig:seleccion-generacion-wdmm-1}


\figura{Protocolo del estado del arte seleccionados por el protocolo híbrido
dinámico en el tiempo de simulación en WDMM para 30 segundos por mensaje generado.}
{
\graficoProtocolosTiempo
{\% de nodos que han seleccionado el protocolo}
{Tiempo de simulación (segundos)}
{data/protocolos/PERSONA_syw_routers_true_nodos_wdmm30.txt}
{data/protocolos/PERSONA_syf_routers_true_nodos_wdmm30.txt}
{data/protocolos/PERSONA_epidemic_routers_true_nodos_wdmm30.txt}
{data/protocolos/PERSONA_prophet_routers_true_nodos_wdmm30.txt}
{data/protocolos/PERSONA_maxprop_routers_true_nodos_wdmm30.txt}
}{fig:seleccion-generacion-wdmm-30}


\figura{Protocolo del estado del arte seleccionados por el protocolo híbrido
dinámico en el tiempo de simulación en WDMM para 120 segundos por mensaje generado.}
{
\graficoProtocolosTiempo
{\% de nodos que han seleccionado el protocolo}
{Tiempo de simulación (segundos)}
{data/protocolos/PERSONA_syw_routers_true_nodos_wdmm120.txt}
{data/protocolos/PERSONA_syf_routers_true_nodos_wdmm120.txt}
{data/protocolos/PERSONA_epidemic_routers_true_nodos_wdmm120.txt}
{data/protocolos/PERSONA_prophet_routers_true_nodos_wdmm120.txt}
{data/protocolos/PERSONA_maxprop_routers_true_nodos_wdmm120.txt}
}{fig:seleccion-generacion-wdmm-120}



La \ref{fig:seleccion-generacion-wdmm-1}, la
\ref{fig:seleccion-generacion-wdmm-30} y la
\ref{fig:seleccion-generacion-wdmm-120} muestran el porcentaje de nodos que han
seleccionado un protocolo en particular del estado del arte durante la
simulación.  Se puede ver que para un mensaje generado cada $1$ segundo
(\ref{fig:seleccion-generacion-wdmm-1}) el protocolo es capaz de decidirse por
un protocolo y seleccionarlo, incluso seleccionado más de uno: \syf{} y \syw. En
el caso de mensajes generados cada $30$ segundos, la situación es completamente
diferente, en la \ref{fig:seleccion-generacion-wdmm-30}, a lo largo de la
simulación no hay ningún consenso por parte del protocolo híbrido dinámico para
seleccionar cual podría ser un buen protocolo del estado del arte en esa
situación. Lo mismo ocurre para la generación de un mensaje cada 120 segundos en
la \ref{fig:seleccion-generacion-wdmm-120}. Entonces, el protocolo tiene un mal
rendimiento debido a que no es capaz de utilizar la información que tiene para
decidirse por un protocolo del estado del arte.



La información utilizada dentro del protocolo híbrido dinámico se obtiene al
seleccionar un nodo persona de la red para extraerle las métricas. La
\ref{fig:delivery-generacion-wdmm-1}, \ref{fig:delivery-generacion-wdmm-30} y
\ref{fig:delivery-generacion-wdmm-120} muestran como el protocolo dinámico
estima la tasa de entrega en WDMM, mientras que en la
\ref{fig:delivery-generacion-pdm-1}, la \ref{fig:delivery-generacion-pdm-30} y
la \ref{fig:delivery-generacion-pdm-120} se muestran los valores estimados de la
tasa de entrega en el escenario de movilidad PDM.



\figura{Tasa de entrega estimada por el protocolo híbrido dinámico en el tiempo
de simulación en WDMM para 1 segundo por mensaje generado.}
{
\graficoEstimacio
{Tasa de entrega estimada}
{Tiempo de simulación (segundos)}
{data/protocolos/delivery_wdmm_1syw.txt}
{data/protocolos/delivery_wdmm_1syf.txt}
{data/protocolos/delivery_wdmm_1epidemic.txt}
{data/protocolos/delivery_wdmm_1prophet.txt}
{data/protocolos/delivery_wdmm_1maxprop.txt}
}{fig:delivery-generacion-wdmm-1}


\figura{Tasa de entrega estimada por el protocolo híbrido dinámico en el tiempo
de simulación en PDM para 1 segundo por mensaje generado.}
{
\graficoEstimacio
{Tasa de entrega estimada}
{Tiempo de simulación (segundos)}
{data/protocolos/delivery_pdm_1syw.txt}
{data/protocolos/delivery_pdm_1syf.txt}
{data/protocolos/delivery_pdm_1epidemic.txt}
{data/protocolos/delivery_pdm_1prophet.txt}
{data/protocolos/delivery_pdm_1maxprop.txt}
}{fig:delivery-generacion-pdm-1}


\figura{Tasa de entrega estimada por el protocolo híbrido dinámico en el tiempo
de simulación en WDMM para 30 segundos por mensaje generado.}
{
\graficoEstimacio
{Tasa de entrega estimada}
{Tiempo de simulación (segundos)}
{data/protocolos/delivery_wdmm_30syw.txt}
{data/protocolos/delivery_wdmm_30syf.txt}
{data/protocolos/delivery_wdmm_30epidemic.txt}
{data/protocolos/delivery_wdmm_30prophet.txt}
{data/protocolos/delivery_wdmm_30maxprop.txt}
}{fig:delivery-generacion-wdmm-30}


\figura{Tasa de entrega estimada por el protocolo híbrido dinámico en el tiempo
de simulación en PDM para 30 segundos por mensaje generado.}
{
\graficoEstimacio
{Tasa de entrega estimada}
{Tiempo de simulación (segundos)}
{data/protocolos/delivery_pdm_30syw.txt}
{data/protocolos/delivery_pdm_30syf.txt}
{data/protocolos/delivery_pdm_30epidemic.txt}
{data/protocolos/delivery_pdm_30prophet.txt}
{data/protocolos/delivery_pdm_30maxprop.txt}
}{fig:delivery-generacion-pdm-30}


\figura{Tasa de entrega estimada por el protocolo híbrido dinámico en el tiempo
de simulación en WDMM para 120 segundos por mensaje generado.}
{
\graficoEstimacio
{Tasa de entrega estimada}
{Tiempo de simulación (segundos)}
{data/protocolos/delivery_wdmm_120syw.txt}
{data/protocolos/delivery_wdmm_120syf.txt}
{data/protocolos/delivery_wdmm_120epidemic.txt}
{data/protocolos/delivery_wdmm_120prophet.txt}
{data/protocolos/delivery_wdmm_120maxprop.txt}
}{fig:delivery-generacion-wdmm-120}

\figura{Tasa de entrega estimada por el protocolo híbrido dinámico en el tiempo
de simulación en PDM para 120 segundos por mensaje generado.}
{
\graficoEstimacio
{Tasa de entrega estimada}
{Tiempo de simulación (segundos)}
{data/protocolos/delivery_pdm_120syw.txt}
{data/protocolos/delivery_pdm_120syf.txt}
{data/protocolos/delivery_pdm_120epidemic.txt}
{data/protocolos/delivery_pdm_120prophet.txt}
{data/protocolos/delivery_pdm_120maxprop.txt}
}{fig:delivery-generacion-pdm-120}


Si se comparan los gráficos entre ellos para un intervalo de generación de
mensajes, se puede ver que en PDM es más sencillo identificar aquel protocolo
que tiene una mayor tasa de entrega en el escenario, mientras que en las
estimaciones de WDMM, no es tan claro cual es el protocolo del estado del arte
que debería seleccionar el protocolo híbrido dinámico. Las malas estimaciones
están relacionadas con la paradoja explicada en la \ref{sub:metricas}, donde se
requiere de que existan mensajes para poder distribuir la información de 
\textit{piggybacking} para estimar la tasa de entrega. La otra razón de que WDMM
le afecta más una mala estimación de la tasa de entrega que a PDM o VMM es
debido a que se utilizó un valor de $\alpha = 0.8$, lo que quiere decir que para
la toma de decisiones es más importante la tasa de entrega que la cantidad de
copias estimadas.

Otro factor que hay que tomar en cuenta respecto a la tasa de entrega es el
margen de mejora. Si se revisan los gráficos de resultados de la tasa de entrega
de WDMM, se puede ver que no existen grandes diferencias como en PDM y WDMM
entre los diferentes protocolos del estado del arte, limitando las opciones que
tiene un protocolo híbrido dinámico para poder mejorar el desempeño de la red.

Finalmente, la movilidad que existe en WDMM es diferente si se compara con PDM o
VMM, como se puede ver en el \ref{sec:evaluacion-movilidad}. En estos dos
últimos modelos de movilidad existe un solo gran cambio de movimiento que ocurre
al momento de evacuar la ciudad, mientras que en WDMM los cambios son más
sutiles, incluso al moverse desde los hogares al lugar de trabajo, donde se
mueven solamente dentro de las oficinas.  

Estos dos factores, la baja cantidad de mensajes que se encuentran en la
simulación, ya sean creados o en forma de copias, y la diferencia entre los
modelos de movilidad causa que existan problemas en la distribución de las
métricas y contextos.

A pesar de que el protocolo híbrido dinámico no sea capaz de seleccionar un
protocolo adecuado en el escenario de movilidad WDMM, existe una ventaja
respecto a \syf, protocolo que siempre logra tener mejor tasa de entrega y menor
\overhead{} que el protocolo dinámico, y esta ventaja es la latencia de la red.
La latencia de la red es mayor para la mayoría de las pruebas que se realizaron
en WDMM llegando a ser el protocolo más lento para transmitir mensajes en ese
escenario de movilidad.


\subseccion{Latencia y escenarios de desastres}

En el \ref{apendice:dinamico} se pueden ver los gráficos de la latencia de la
red en cada uno de los experimentos anteriores. Si bien, es importante que el
protocolo sea capaz de entregar los mensajes rápidamente, dependiendo del
escenario otras métricas pueden ser más importantes. Para una autoridad es
importante poder entregar mensajes de seguridad o información de manera rápida,
lo que quiere decir que se debe priorizar la latencia y la tasa de entrega para
maximizar la cantidad de usuarios que reciban los mensajes. Para poder soportar
este caso de uso es, es necesaría una modificación al protocolo híbrido dinámico
donde se consideren mensajes con prioridad y se aplique una nueva función
objetivo para que el protocolo híbrido dinámico pueda maximizarla.


Para los experimentos realizados en este trabajo, las fuentes y destinos de los
mensajes podían ser cualquier nodo de la red sin ninguna prioridad que los
hiciera especiales, razón por la cual todos son tratados de la misma forma por
el protocolo híbrido dinámico y no se considera la latencia de la red. De
cualquier forma, las latencias que se presentan en el \ref{apendice:dinamico} se
encuentran dentro de un rango razonable debido a que corresponden a
aproximadamente un $15$\% del tiempo total de simulación en el caso de los
escenarios de desastres y en un $25$\% del tiempo total de simulación en WDMM.





% La falta de información

\seccion{Conclusiones del protocolo híbrido dinámico}

El protocolo híbrido dinámico es una versión del protocolo híbrido estático que
no asume cual es el mejor protocolo para una movilidad y para obtener una
reducción en el consumo de energía. El protocolo híbrido dinámico para
escenarios de desastres es capaz de lograr un \textit{trade-off} aceptable entre
la tasa de entrega y el \overhead{} que permite la construcción de redes DTN más
longevas, especialmente en escenarios de desastres.

A pesar de que en el modelo de movilidad \textit{Working Day Mobility Model}
los resultados no presentan una mejora respecto a otros protocolos del estado
del arte, en \textit{Valparaíso Mobility Model} si hay una ventaja de utilizar
el nuevo protocolo dinámico respecto a las soluciones anteriores, lo que
demuestra que en el caso de un desastre natural en una ciudad chilena el nuevo
protocolo puede reducir el consumo de energía.

Una posible mejora en caso que los mensajes de la red utilicen prioridades,
un protocolo híbrido dinámico podría tener más de una función objetivo, una por
cada nivel de prioridad de manera tal de poder aplicar el protocolo más adecuado
para el tipo de mensaje. Un mensaje prioritario podría tener una función objetivo
que minimice la latencia sin preocuparse del \overhead{} generado.


Además, el protocolo general del protocolo híbrido dinámico general entrega un
\textit{framework} para que se implementen otros protocolos con otras funciones
objetivos para tratar de maximizar o minimizar otras métricas de redes con
propósitos diferentes a los de escenarios de desastres.

