

Debido a cada uno de los tipos de movimientos provistos por PDM, es posible
definir protocolos que puedan utilizar estas diferencias en la decisión de
encaminaminamiento. Una clasificación que se ha realizado ha sido la que se
muestra en la \ref{tabla:clasificacion}. Aquellos nodos que tienen una alta
movilidad, es decir, que recorren grandes extensiones de la ciudad en un corto
tiempo son llamados exploradores mientras que aquellos que tienen una movilidad
más limitada se llaman explotadores y además se proponen
\cite{paper_evaluacion_nosotros} los protocolos más adecuados para cada
situación.


\tabla{Clasificación de nodos según su movilidad en PDM (Basado en \cite{paper_evaluacion_nosotros})}{
    \begin{tabularx}{\textwidth}{|l|l|X|}
            \hline
            Nodo       & Tipo          & Protocolo Preferido \\ \hline \hline
            Persona/Voluntario & Explotador    & \syw \\ \hline
            Rescatista & Explotador    & \syw, \syf \\ \hline
            Vehículo   & Explorador    & \epidemic, \maxprop, \prophet \\ \hline
    \end{tabularx}
}{tabla:clasificacion}
