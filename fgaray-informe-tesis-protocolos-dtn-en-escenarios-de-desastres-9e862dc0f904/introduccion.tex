\seccion{Antecedentes y motivación}


Sistemas como el de alerta temprana de Japón permiten enviar datos sobre
terremotos que estén en proceso mediante la red de telefonía para que la
población pueda moverse a lugares seguros \cite{erika_yamasaki_what_2012}. Este
sistema ha sido replicado en Chile por parte de la ONEMI (Oficina Nacional de
Emergencias del Ministerio del Interior) y SUBTEL (Subsecretaría de
Telecomunicaciones) para alertar sobre los riesgos de \textit{tsunamis}, sismos
de mayor intensidad, erupciones volcánicas e incendios forestales
\cite{gobierno_de_chile_sistema_????, subtel_onemi_2014}.  Sin embargo, este
sistema de alerta temprana requiere de una infraestructura de comunicaciones
para funcionar la cual puede resultar dañada luego de un desastre natural como
terremotos o inundaciones haciendo difícil para las personas del lugar y
autoridades recibir y enviar información que puede resultar de vital importancia
para la toma de decisiones.

Un ejemplo de los problemas que surgen frente a la falta de comunicaciones
ocurrió en el terremoto del 2010 en Chile, donde durante cuatro días la región
del Bío Bío tenía solamente un 27\% de operatividad en telefonía móvil y en
telefonía fija un 45\%  \cite{subtel_reporte_2015} haciendo las comunicaciones
inviables en esas zonas durante un largo periodo de tiempo.


Una solución que se aplica en este contexto son las \textit{Delay Tolerant
Networks} (DTNs) o redes tolerantes a disrupciones las cuales utilizan el
movimiento de los nodos para encontrar posibles rutas en la red, esto se logra
mediante el paradigma \textit{store-and-forward}  que consiste en almacenar los
mensajes y utilizar el movimiento de las personas (nodos) para hacer llegar los
mensajes a sus destinos \cite{fall_delay-tolerant_2003}. Cuando se aplica este
tipo de redes en el contexto de desastres, generalmente son dispositivos móviles
como teléfonos celulares los que ejecutan los distintos protocolos DTN los que
tienen la responsabilidad de decidir si transmitir o no un mensaje a un nodo
vecino.  Dado que el mayor consumo de energía ocurre mediante la transmisión de
datos entre nodos y a que esta es un recurso escaso que limita la movilidad  de
los nodos en el contexto de desastre, es importante encontrar una manera de
reducir este consumo para que sea factible poder aplicar estas redes en estos
escenarios.

En el contexto del grupo de investigación RESPOND del Departamento de Ingeniería
Informática de la Universidad de Santiago de Chile, existe un proyecto de
detección de necesidades de las personas el cual requiere tanto de comunicación
de dos vías como de robustez frente a un posible desastre en el país. Además,
este trabajo de investigación corresponder a un proyecto DICYT.



\seccion{Descripción del problema}

Dado que este trabajo nace del alto gasto energético que surge al aplicar DTNs y
a que la energía es un recurso escaso en escenarios de desastres, es importante
tratar de encontrar formas en que un usuario participante de la red pueda enviar
y recibir mensajes sin necesidad de recargar la batería de su dispositivo en
espacios cortos de tiempo, lo que puede resultar imposible si el desastre afecta
la infraestructura de energía. Además, dado que en un escenario de desastre
participan usuarios que tienen distintos tipos de movimiento y dispositivos, se
hace necesario encontrar una manera de explotar esta situación para mejorar el
desempeño de la red.  Es por esta razón que nacen los siguientes problemas:


\begin{itemize}
    \item ¿Cómo reducir el consumo de energía en una red DTN utilizando los
      protocolos del estado del arte como: \protocolos?
    \item ¿Cómo afectan los distintos patrones de movilidad que existen en un
    escenario de desastre al desempeño de los protocolos DTN?  
    \item ¿Cómo comunicar a las personas en un escenario de desastre?
\end{itemize}



\seccion{Solución propuesta}

Para dar solución al problema planteado se propone la creación de un protocolo
híbrido dinámico que pueda generar y recolectar información del desempeño de la
red para poder actuar en el mismo momento de la ejecución y poder seleccionar el
protocolo más adecuado para reducir el consumo de energía de acuerdo a su
movilidad interna. Este protocolo va a ser híbrido, dado que mezcla los
algoritmos de encaminamiento protocolos del estado del arte en uno solo.





\seccion{Objetivos y alcances del proyecto}



\subseccion{Objetivo general}
{
Crear un protocolo de redes tolerante a disrupciones para dispositivos móviles
en escenarios de desastre. 
}

\subseccion{Objetivos específicos}
{
\begin{enumerate}
    \item Elaborar un estado del arte sobre redes tolerantes a disrupciones,
      modelos de movilidad y modelos de energía para dispositivos móviles.
    \item Adaptar el modelo de movilidad \textit{Post-Disaster Mobility Model}
      (PDM) \cite{uddin_post-disaster_2009} a Chile.
    \item \label{objetivo-diseno} Diseñar un protocolo para redes tolerantes a
      disrupciones que se adapte las decisiones de enrutamiento en base a la
      movilidad de los dispositivos móviles participantes.  
    \item Implementar el protocolo en el simulador de redes tolerantes a
      disrupciones \textit{The One}.
    \item Evaluar el rendimiento del protocolo híbrido respecto a los protocolos
      del estado del arte en diferentes modelos de movilidad.
\end{enumerate}
}


\newpage
\subseccion{Alcances}


\begin{itemize}
    \item Debido a las limitaciones de probar un nuevo protocolo en un ambiente
    real, solamente se van a obtener resultados en forma de simulaciones sin
    implementarlo en dispositivos reales.
    \item El modelo de energía a utilizar es el propuesto por Martín-Campillo et
      al. \cite{martin-campillo_energy-efficient_2012} el cual permite estimar
      el consumo de energía de un dispositivo dentro de un simulador en base a
      la cantidad de copias de mensajes efectuadas.
    \item La adaptación del modelo de movilidad PDM a Chile no está validada con
      trazas reales de una ciudad chilena.
    \item Debido a los altos tiempos de simulación, no es posible realizar
      experimentos con todos los posibles valores de las variables, por lo que
      solo se utiliza aquellos valores más interesantes.
\end{itemize}




%%% FIN OBJETIVOS


\seccion{Metodologías y herramientas utilizadas}


\subseccion{Metodología}


Como metodología de desarrollo se va a utilizar una mezcla entre el método
científico y una metodología de desarrollo de software y testing
\cite{hernandez_sampieri_metodologiinvestigacion_2010}:

La hipótesis de este trabajo es: "Utilizando información sobre la movilidad de
un nodo y el desempeño de los protocolos, se puede seleccionar el protocolo más
adecuado para reducir el consumo de energía en comparación con la utilización de
un solo protocolo en la red". La metodología tiene los siguientes pasos:



\begin{enumerate}
    \item Plantear el problema de investigación: Establecimiento de los
      objetivos de la investigación.
    \item Elaborar un estado del arte y marco teórico.
    \item Formulación/Reformulación de la hipótesis: Averiguar cuales son los
        factores que podrían reducir el consumo de energía en el protocolo.
    \item Probar la hipótesis mediante experimentación:
        \begin{enumerate}
            \item Definir los cambios/requisitos del software de acuerdo a la
                hipótesis.
            \item Crear pruebas unitarias a partir de los cambios requisitos,
                probarlas y esperar a que fallen. Esto es aplicar Test-Driven
                Development \cite{ble_jurado_diseno_2010} (TDD) al diseño del software.
            \item Implementar el software para que pasen las pruebas unitarias.
            \item Realizar las simulaciones.
        \end{enumerate}
    \item Interpretar los datos del experimento y sacar conclusiones: Comparar
        con simulaciones de otros protocolos.
    \item Presentar los resultados.
\end{enumerate}



La parte iterativa de esta metodología es entre la etapa 3 y la 5, si se
encuentra que los resultados no son lo que se esperaba de la hipótesis, entonces
se debe volver al paso 3 para volver a reformular las formas de reducir el
consumo de energía. Se terminará cuando el protocolo creado implemente una
métrica basada en la energía y en los vecinos que mejore el consumo de energía
comparado con los protocolos originales.


Como estrategia empírica se va a utilizar la técnica de la experimentación
\cite{wohlin_experimentation_2012} donde se manipula una variable  del conjunto
(cantidad de nodos, tamaño de los mensajes, cantidad de mensajes en la
simulación, destinos de los mensajes, tamaño de \textit{buffers} y velocidad de
comunicación) estudiada. Estos experimentos son comúnmente efectuado en
ambientes controlados o, como en este caso, un simulador.



\subseccion{Herramientas de desarrollo}

El protocolo fue desarrollado y probado utilizando el simulador \theone{}
\cite{keranen_one_2009} versión 1.5.1 RC2 en los lenguajes de programación Java
1.8 \footnote{https://www.java.com/es/} y Scala 2.11
\footnote{https://www.scala-lang.org/}. 


Como editor de texto tanto para el protocolo como para el informe se utilizó
\textit{VIM} 8.0 \footnote{http://www.vim.org/} y como sistema de composición de
texto de este documento se utilizó \LaTeX
\footnote{https://www.latex-project.org/} con el motor XeLaTeX
\footnote{http://xetex.sourceforge.net/}.

Para la generación de gráficos se utilizó \textit{PGFPlots}
\footnote{http://pgfplots.sourceforge.net/} en su versión 1.13 y para la
generación de figuras y diagramas Ti$k$Z 3.0.1
\footnote{https://sourceforge.net/projects/pgf/}, ambos para \LaTeX. Diagramas
adicionales fueron creados utilizando \textit{Inkscape}
\footnote{https://inkscape.org/es/}. Para el procesamiento de los resultados
obtenidos de las simulaciones se utilizó Python 2.7
\footnote{https://www.python.org} y GNU Awk 4.1
\footnote{https://www.gnu.org/software/gawk/}.

Las simulaciones se ejecutaron en un servidor GNU/Linux Ubuntu con 32 núcleos y
125 GB de RAM que permite ejecutar las simulaciones en paralelo.


\seccion{Organización del documento}

En los siguientes capítulos de esta tesis se presentan los resultados obtenidos
respecto a la reducción del consumo de energía de protocolos DTN en escenarios
de desastres, así como tambien problemas de seguridad que aparecen en este tipo
de redes y un nuevo modelo de movilidad basado en Chile para la evaluación del
protocolo creado.

En el \ref{chp:arte} se presenta un estado del arte y marco teórico abarcando
conceptos de DTN, consumo de energía, modelos de movilidad y seguridad
explicando aquellas publicaciones y soluciones relacionadas con la propuesta en
este trabajo. El \ref{chp:pasado} presenta un protocolo híbrido estático, es
decir, cada nodo puede ejecutar más de un protocolo pero no puede cambiarlo ni
reaccionar ante cambios en la red. Esta es una solución preliminar que se
utiliza como base para plantar la hipótesis y para la creación del protocolo
híbrido dinámico.

Debido a que PDM no fue diseñado para simular un ambiente como el chileno, con
sus propias ciudades y desastres naturales, es que es necesario adaptarlo al
contexto nacional. En el \ref{chp:movilidad} se presenta un nuevo modelo de
movilidad basado en PDM llamado \textit{Valparaíso Mobility Model} que modela la
situación luego de un terremoto en una ciudad costera como Valparaíso donde las
personas deben evacuar hacia las alturas. Este modelo de movilidad se utiliza
para evaluar el desempeño del protocolo híbrido dinámico.

Utilizando el protocolo híbrido estático como base, se presenta en el
\ref{chp:dinamico} un protocolo híbrido dinámico que puede reaccionar ante
cambios en la red y selecionar el protocolo del estado del arte más adecuado
para reducir el consumo de energía relacionado con la comunicación.

Debido a que las DTNs son redes abiertas donde cualquier dispositivo puede
participar es que se debe abordar el problema de la seguridad. El
\ref{chp:seguridad} se trata el problema de los ataques de denegación de
servicio y como mitigar su efecto en escenarios de desastres.

Finalmente, en el \ref{chp:conclusiones} se presentan las conclusiones de las
soluciones propuestas presentando un análisis final de los resultados
experimentales y trabajo futuro que puede ser abordado como continuación.





\seccion{Otros resultados}

Partes de este trabajo se presentaron en conferencias internacionales,
específicamente la propuesta de un protocolo híbrido estático DTN fue presentada
en una publicacion titulada \textit{Mobility-aware DTN protocols for
post-disaster scenarios} \cite{paper_evaluacion_nosotros} y la capa de seguridad
fue propuesta en el trabajo titulado \textit{Reliable Routing Protocol for Delay
Tolerant Networks} \cite{DBLP:conf/icpads/GarayRH15}.

Ambos trabajos fueron co-creados por el autor de esta tesis.
