% Planeamiento del problema
% Método utilizado
% Principales resultados y conclusiones
% Señalar si la tesis forma parte de un proyecto interno o externo.
% 300 palabras


Recolectar y diseminar informaci\'on luego de un desastre puede resultar
complejo debido a los posibles fallos en la infraestructura de comunicaci\'on y
al comportamiento din\'amico de las personas. Las soluciones actuales explotan los
dispositivos m\'oviles haciendo uso de protocolos con tolerancia a disrupciones
(DTNs) para obtener y transmitir informaci\'on en presencia de conectividad
intermitente. Sin embargo, la mayor\'ia de los protocolos de estas redes son muy
costosos en t\'erminos de energ\'ia, un recurso escaso y valioso cuando se trata
de dispositivos m\'oviles en un escenario de desastre. Uno de los factores que
incrementa el consumo energético es la comunicación de mensajes, que puede ser
reducida usando más de un protocolo en una misma red.


En esta tesis se proponen dos nuevos protocolos para DTNs en escenarios de
desastres con el objetivo de reducir el consumo de energía de los nodos mediante
el uso de más de un protocolo en la red, es decir, cada nodo ejecuta distintas
reglas de encaminamiento dependiendo de su movilidad y desempeño de la red,
permitiendo la optimización del uso de los recursos disponibles. Un tercer
protocolo DTN también es propuesto para mitigar posibles ataques de denegación
de servicio mitigando el efecto del ataque a la red.

Utilizando simulaciones para comparar los protocolos creados con los del estado
del arte, se encuentra que los nuevos protocolos presentan una reducción
significativa en el consumo de energía. Para poder analizar como estos
protocolos se comportan en una ciudad de Chile, se propone, además, un nuevo
modelo de movilidad basado en la ciudad de Valparaíso utilizando información
entregada por la Oficina Nacional de Emergencias (ONEMI).


Este trabajo forma parte del proyecto DICYT-USACH 061419RO \textit{Recolección
eficiente de datos para dispositivos móviles}.
